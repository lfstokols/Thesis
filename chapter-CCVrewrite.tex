\section{Isoperimetric Inequality for low-order diffusion} \label{ch:CCVrewrite}

%\documentclass[11pt]{article}
%\usepackage{amsmath,amssymb,amsthm}
%\usepackage{graphicx,mathtools,mathrsfs}
%\usepackage[margin=1in]{geometry}
%\usepackage{fancyhdr}
%\setlength{\parindent}{0pt}
%\setlength{\parskip}{5pt plus 1pt}
%\setlength{\headheight}{13.6pt}
%\newcommand\question[2]{\vspace{.25in}\hrule\textbf{#1: #2}\vspace{.5em}\hrule\vspace{.10in}}
%\renewcommand\part[1]{\vspace{.10in}\textbf{(#1)}}
%\pagestyle{fancyplain}
%
%
%%-------------------------------------------<commands>--------------------------------------------------------
%
%\newcommand{\R}{\mathbb{R}}
%\newcommand{\N}{\mathbb{N}}
%\newcommand{\Z}{\mathbb{Z}}
%\newcommand{\Q}{\mathbb{Q}}
%\newcommand{\C}{\mathbb{C}}
%\newcommand{\Prj}{\mathbb{P}}
%\newcommand{\F}{\mathcal{F}}
%\newcommand{\A}{\mathbb{A}}
%\newcommand{\T}{\mathbb{T}}
%\newcommand{\E}{\mathbb{E}}
%
%\newcommand{\floor}[1]{\left\lfloor #1 \right\rfloor}
%\newcommand{\ceil}[1]{\left\lceil #1 \right\rceil}
%\newcommand{\chevron}[1]{\langle #1 \rangle}
%\newcommand{\norm}[1]{\left\lVert#1\right\rVert}
%\newcommand{\paren}[1]{\left( #1 \right)}
%\newcommand{\abs}[1]{\left\lvert #1 \right\rvert}
%
%\DeclareMathOperator{\id}{id}
%\DeclareMathOperator{\convex}{conv}
%\DeclareMathOperator{\image}{Im}
%\DeclareMathOperator{\im}{Im}
%\DeclareMathOperator{\coker}{coker}
%\DeclareMathOperator{\supp}{supp}
%\DeclareMathOperator{\trace}{tr}
%\DeclareMathOperator{\lspan}{span}
%\DeclareMathOperator{\conv}{conv} % stands for conv, as in convex hull
%\DeclareMathOperator{\Int}{int} % stands for int, as in interior of a set
%\DeclareMathOperator{\sign}{sign}
%\DeclareMathOperator{\ran}{ran}
%\DeclareMathOperator{\rank}{rank}
%%\DeclareMathOperator{\dim}{dim}
%\newcommand{\dom}{\operatorname{dom}}
%\newcommand{\cod}{\operatorname{cod}}
%\newcommand{\Hom}{\operatorname{hom}}
%\newcommand{\Ob}{\operatorname{Ob}}
%\newcommand{\cl}{\operatorname{cl}}
%
%\newcommand{\del}{\partial}
%\newcommand{\pvec}[2]{\frac{\partial #1}{\partial #2}}
%\newcommand{\grad}{\nabla}
%\newcommand{\D}[1]{\frac{d}{d #1}}
%
%\newcommand{\into}{\hookrightarrow}
%\newcommand{\onto}{\twoheadrightarrow}
%\newcommand{\isom}{\cong}
%\newcommand{\rest}{{\upharpoonright}}
%\newcommand{\weakly}{\rightharpoonup}
%
%
%\newcommand{\ith}{^\mathrm{th}}
%\newcommand{\n}{^{-1}}
%
%\newcount\colveccount
%\newcommand*\colvec[1]{
%        \global\colveccount#1
%        \begin{bmatrix}
%        \colvecnext
%}
%\def\colvecnext#1{
%        #1
%        \global\advance\colveccount-1
%        \ifnum\colveccount>0
%                \\
%                \expandafter\colvecnext
%        \else
%                \end{bmatrix}
%        \fi
%}
%-------------------------------------------</commands>--------------------------------------------------------


%\begin{document}\raggedright
%
%\author{Logan Stokols}
%\title{CCV 2nd De Giorgi rewrite}
%\maketitle

The proof of the isoperimetric inequality outlined in section~\ref{sec:intro-DG outline} relies crucially on the fact that functions with finite $H^1$ norm cannot have jump discontinuities.  For the fractional heat equation
\begin{equation} \label{eq:CCV-main}
\del_t u + (-\Laplace)^{s/2} u = 0,
\end{equation}
this proof breaks down when $s < 1$.  This is because the natural energy inequality for this equation shows that solutions are in $H^{s/2}$, and elements of $H^{s/2}$ can have jump discontinuities whenever $s < 1$.  For example, $\indic{|x|<1} \in H^{1/4}$.  However, solutions to these low-order heat equations are in fact H\"{o}lder continuous.  This is shown, for example, by Caffarelli, Chan, and Vasseur in \cite{CaChVa.nio} using the De Giorgi method.  The technique they use is based instead on the nonlocality of $(-\Laplace)^{s/2}$.  In fact, the technique will work for any $s \in (0,2)$.  

This technique originated in the work of Bass and Kassmann in \cite{BaKa}, before being used in \cite{CaChVa.nio} and later in \cite{St.hypo} (which is contained in the present work as chapter~\ref{ch:kinetic}, c.f. proposition \eqref{thm:DG2.kinetic} step 3).  

Because this technique has the potential to be utilized in a wide variety of circumstances, we will present two versions of the method in its simplest possible context.  

\vskip.3cm

Recall that
\begin{equation} \label{CCV-coercivity for K} 
(-\Laplace)^{s/2} u(x) := \int c_{d,s} \frac{[u(x)-u(y)]}{|x-y|^{d+s}} \,dy.
\end{equation}

We will use the following notation, modelled after \cite{CaChVa.nio}:
\begin{align*}
B[u,v] &:= \int_{\R^d} \int_{\R^d} K(t,x,y) [u(x)-u(y)] [v(x)-v(y)] \,dxdy, \\
%\psi_1(x) &:= 1 + (|x|^{s/2} - 1)_+, \\
\psi_\lambda(x) &:= \paren{|x| - \lambda^{-4/s} }_+^{s/4} - 1)_+, \\
F(x) &:= \sup(-1, \inf(0, |x|^2 - 9), \\
\phi_0 &= 1 + \psi_\lambda + F, \\
\phi_1 &= 1 + \psi_\lambda + \lambda F.
\end{align*}
Note that 
\[ 0 \leq \psi_{\lambda_1}(x) \leq \psi_{\lambda_2}(x) \]
for $\lambda_1 < \lambda_2$, and for $\lambda$ sufficiently small we have $\psi_\lambda(x) \equiv 0$ for $|x| \leq 3$.  The function $F$ is non-positive, compactly supported in $B_3(0)$, and equal to $-1$ in $B_2$.  Therefore $\phi_0$ vanishes on $B_2(0)$.  

It is proven in \cite{CaChVa.nio} that for $u$ solving \eqref{eq:CCV-main} with $u(x) \leq \psi_1(x)$ for all $|x|\leq 3$, we have
\begin{equation} \label{CCV-energy inequality generic}
\ddt \int (u - \psi)_+^2 + \norm{(u-\psi)_+}^2_{H^{s/2}(\R^d)} - B[(u-\psi)_+,(u-\psi)_-] \leq C \bracket{\int (u-\psi)_+^2 + \iint \indic{u-\psi_1 > 0}(x)[\psi(y)-\psi(x)]^2 \,dxdy }
\end{equation}

The following lemma is equivalent to Lemma 4.1 in \cite{CaChVa.nio}.  
\begin{lemma}[Isoperimetric Inequality for \cite{CaChVa.nio}]
Let $\delta>0$ be a constant, $s \in (0,2)$ and $d \in \N$.  Then, there exists $\gamma > 0$, and $\lambda \in (0,1)$, depending only on $\delta$, $d$, and $s$, such that for any solution $u:[-3,0] \times \R^d \to \R$ of \eqref{eq:CCV-main} satisfying
\begin{equation} \label{CCV-pointwise bound}
u(t,x) \leq 1 + \psi_\lambda(x) \qquad \textrm{on } [-3,0]\times\R^d, 
\end{equation}
\[ \abs{\{u < \phi_0\} \cap (-3,-2)\times B_1} \geq |B_1|, \]
and
\[ \abs{\{u > \phi_1\} \cap (-2,0)\times \R^d} \geq \delta, \]
we have
\[ \abs{\{\phi_0 < u < \phi_1\} \cap (-3,0)\times \R^d} \geq \gamma. \]
\end{lemma}
%Note that this statement is slightly reformulated from that found in \cite{CaChVa.nio} for simpler notation in the following proof.  Where they differ, the formulation given here makes a slightly stronger claim than the original.  

The proof follows three steps: first we consider, at each time-slice, the sets where $u$ is above $\phi_1$, below $\phi_0$, and between the two.  Second we use an intermediate cutoff between $\phi_0$ and $\phi_1$, together with the energy inequality, to show that the derivative of a certain function $H$ is bounded above, and also the energy $B[u_+,u_-]$ corresponding to the interaction between small and large values of $u$ is bounded above.  Third, we combine these facts to find a large set of times where $u$ is between $\phi_0$ and $\phi_1$.  

\begin{proof}
Define the following set-valued functions:
\begin{align*}
A_0(t) &= \{x\in B_1: u(x,t) \leq \phi_0(x)\} \\
A_{1/2}(t) &= \{x\in \R^d: \phi_0 < u(x,t) < \phi_1 \} \\
A_1(t) &= \{x\in \R^d: \phi_1 \leq u(x,t)\}.
\end{align*}
Clearly $|A_0(t)| + |A_{1/2}(t)| + |A_1(t)| \geq |B_1|$ for all $t$.  By assumption, we must have $\int_{-3}^{-2} |A_0(t)|dt \geq |B_1|$ and $\int_{-2}^0 |A_1(t)|dt \geq \delta$.  From the assumption \eqref{CCV-pointwise bound} we know that $A_{1/2}(t), A_1(t) \subseteq B_3$ and of course $A_0(t) \subseteq B_1$ for all $t \in [-3,0]$.  

By Chebyshev's inequality, for any $\alpha$ sufficiently small we have
\begin{equation} \label{CCV-big 99percent of the time}\begin{aligned}
\abs{[-3,-2] \cap \{|A_0| > \alpha\}} &\geq 0.99, \\
|[-2,0] \cap \{|A_1| > \alpha\}| &\geq 1.99.
\end{aligned} \end{equation}
Choose one such $\alpha$, small enough that
\begin{equation} \label{CCV-asssumption on gamma_0}
7 \alpha< |B_1|.  
\end{equation}
Without loss of generality, we can assume
\begin{equation} \label{CCV-big 99percent M}
\abs{[-3,0] \cap \{|A_{1/2} > \alpha \}} \leq .01 
\end{equation}
because otherwise, we could take $\gamma = .01 \alpha$ and the proof would be complete.  

\vskip.3cm

Define 
\[ \phi_{1/2}:= 1 + \psi_\lambda + 2 \lambda F. \]
This is an intermidiary cutoff $\phi_0 \leq \phi_{1/2} \leq \phi_1$.  By the assumption \eqref{CCV-pointwise bound}, we know that 
\[ \paren{u-\phi_{1/2}}_+ \leq -2\lambda F. \]

One can calculate that
\[ \iint \indic{u > \phi_{1/2}}(x) [\phi_{1/2}(y)-\phi_{1/2}(x)]^2 K(t,x,y) \]
\[ \leq C \iint \indic{B_3}(x) [\psi_\lambda(y)-\psi_\lambda(x)]^2 K(t,x,y) + \iint [2\lambda F(x)-2 \lambda F(y)]^2 K(t,x,y) \leq C \lambda^2 \]
and that
\begin{align*} 
\int \paren{u-\phi_{1/2}}_+^2 &\leq C \int (2\lambda F)^2
\\ &\leq C \lambda^2.
\end{align*}

Therefore, by the energy inequality \eqref{CCV-energy inequality generic}, 
\begin{equation} \label{CCV-derivative bound} \ddt \int \paren{u-\phi_{1/2}}_+ \,dx \leq C \lambda^2 \end{equation}
and
\begin{equation} \label{CCV-energy inequality lambda^2}
- \int_{-3}^0 B\bracket{\paren{u-\phi_{1/2}}_+, \paren{u-\phi_{1/2}}_-} \,dt \leq C \lambda^2.
\end{equation}

On the other hand, for any fixed $t \in [-3,0]$, the good term is bounded below
\begin{align*} 
-B\bracket{\paren{u-\phi_{1/2}}_+, \paren{u-\phi-{1/2}}_-} &= 2 \iint \paren{u-\phi_{1/2}}_+(x) \paren{u-\phi_{1/2}}_-(y) K(x,y)
\\ & \geq \iint\limits_{x,y \in B_3} \paren{u-\phi_{1/2}}_+(x) \paren{u-\phi_{1/2}}_-(y) 6^{-d-2s}
\\ &= C \int_{B_3} \paren{u-\phi_{1/2}}_+ \,dx \int_{B_3} \paren{u-\phi_{1/2}}_- \,dx
\\ &\geq C \int_{A_1} \lambda F \,dx \int_{A_0} (1-\lambda) F \,dx
\\ &\geq C \lambda \int_{A_1} F \,dx \int_{A_0} F \,dx.
\end{align*}

Therefore, combining this with \eqref{CCV-energy inequality lambda^2} and dividing by $\lambda$, we see that
\begin{equation} \label{CCV-integlar form of interaction bound} \int_{-3}^0 \paren{ \int_{A_1} F \,dx \int_{A_0} F \,dx }\,dt \leq C \lambda. \end{equation}
Note that for any $E \subseteq B_3$, $\int_E F \geq f(|E|)$ for some monotone-increasing function $f$.%, with $f(0) = 0$.  
Therefore the above inequality shows that $|A_0(t)|$ and $|A_1(t)|$ cannot both be large at the same time.  
%In particular, there exists a quantity $e$ such that, if $T(t) \geq \tau$ and $B(t) \geq \beta$, then 
This is an integral bound, however, not a pointwise-in-time bound, so there may exist a singular set
\[ S := \{ t \in [-3,0]: |A_1|>\alpha \textrm{ and } A_0 > \alpha\}. \]
However, for $\alpha$ fixed, because of \eqref{CCV-integlar form of interaction bound}, Chebyshev's inequality, and the fact that $f$ is monotone increasing, we can make this singular set arbitrarily small by taking $\lambda$ very small.  In little-o notation,
\begin{equation} \label{CCV-small measure of overlap} 
|S| = o(\lambda).
\end{equation}


\vskip.3cm

Now we consider the function
\[ H(t) := \int \paren{u(t,x) - \phi_{1/2}(x)}_+^2 \,dx. \]
Recall from \eqref{CCV-derivative bound} that $H'(t) \leq C \lambda^2$.  

By dividing up the support of $\paren{u-\phi_{1/2}}_+$ into the regions $A_0$, $A_{1/2}$, $A_1$, and $\{u \leq \phi_0\} \setminus B_1$, we see that
%\[ \int_{u \leq \phi_0} \paren{u-\phi_{1/2}}_+ = 0 \]
%\[ 0 \leq \int_{A_{1/2}} \paren{u-\phi_{1/2}}_+ \leq \lambda^2 |A_{1/2}|, \]
%\[ \lambda^2 \bracket{|B_1| - |A_{1/2}| - |A_0|} \leq \lambda^2 |A_1| \leq \int_{\phi_1 \leq u} \paren{u-\phi_{1/2}}_+ \leq (2\lambda)^2 |A_1|. \]
%
%Thus 
\[ \lambda^2[|B_1| - |A_0| - |A_{1/2}|] \leq H(t) \leq \lambda^2 [4 |A_1| + |A_{1/2}|]. \]

%Let $\gamma_0 > 0$ be a small constant, and assume $\tau$, $\beta$, and $\gamma_0$ are small enough that $4\tau + \gamma_0 < |B_1|-\beta-\gamma_0$.  
Then we can divide $[-3,0]$ into three sets,
\begin{align*}
H_0 &:= \{t: |A_1(t)|, |A_{1/2}(t)| \leq \alpha \} \subseteq \{t: H(t) \leq \lambda^2 5 \alpha \}, \\
H_1 &:= \{t: |A_0(t)|, |A_{1/2}(t)| \leq \alpha \} \subseteq \{t: \lambda^2[|B_1|-2\alpha] \leq H(t) \}, \\
H_{1/2} &:= \{t: \lambda^2 5 \alpha < H(t) < \lambda^2[|B_1|-2\alpha] \} \subseteq \{t: |A_{1/2}(t)| > \alpha \textrm{ or } \paren{|A_1(t)|, |A_0(t)| > \alpha} \}.
\end{align*}
Note that $\lambda^2 5 \alpha < \lambda^2[|B_1|-2\alpha]$ by \eqref{CCV-asssumption on gamma_0}.  

%The set $H_{1/2}$ cannot have large measure.  
%\[ H_{1/2} \subseteq \]

Our goal now is to show that $H_{1/2}$ has large measure.  To prove this, we will use the fact that $H'$ is bounded and $H_0$, $H_1$ are non-empty.  

The set $H_0 \cap [-3,-2]$ is nonempty, because 
\[ H_0 \supseteq [-3,-2] \cap \{A_0 > \alpha \} \cap S^\complement \cap \{A_{1/2} > \alpha \}^\complement. \]
But by the assumptions \eqref{CCV-big 99percent of the time}, \eqref{CCV-big 99percent M}, and \eqref{CCV-small measure of overlap} we can say $|H_0| \geq .99-.01-|S| > 0$ for $\lambda$ sufficiently small.  
We argue similarly for 
\[ H_1 \supseteq [-2,0] \cap \{A_1 > \alpha \} \cap S^\complement \cap \{A_{1/2} > \alpha \}^\complement \]
that $H_1 \cap [-2,0] \neq \emptyset$.  

We have shown that there exist points $s \in H_0$ and $t \in H_1$ such that $s < t$.  Because $H' \leq C \lambda^2$ by \eqref{CCV-derivative bound}, we know by elementary calculus that
\[ \abs{H_{1/2}} \geq \frac{\lambda^2[|B_1|-2\alpha] - \lambda^2 5 \alpha }{\sup H'} \geq \frac{\lambda^2[|B_1| - 7\alpha }{C \lambda^2} = \bar{C}. \]
Note that $\bar{C}$ depends on $\alpha$ but not on $\lambda$.  

Recall that $H_{1/2} \subseteq \{A_{1/2} > \alpha \} \cup S$.  Therefore
\[ \abs{\{A_{1/2} > \alpha\}} \geq \bar{C} - |S|. \]
Because $|S|$ is arbitrarily small by \eqref{CCV-small measure of overlap}, and because
\[ \abs{\{\phi_0 < u < \phi_1\} \cap (-3,0)\times \R^d} = \int_{-3}^0 |A_{1/2}(t)| \,dt \geq \alpha \abs{\{|A_{1/2} > \alpha \}}, \]
the proof is complete.  

\end{proof}








Note that the above proof is constructive, i.e. $\gamma$ and $\lambda$ can in principle be calculated explicitly.  

Compare this proof to the standard De Giorgi argument, applicable when $s > 1$.  

Assume that we have some compactness result of the form 
\[ \{u \textrm{ solving } \eqref{eq:CCV-main} \textrm{ s.t. } u(t,x) \leq 1 + \psi_{1/3} \forall (t,x) \in [-4,0]\times\R^d \} \Subset L^2([-3,0]\times B_3). \]
Such a result is usually utilized in a proof of an isoperimetric inequality for the De Giorgi method, though we did not use such a result in the above proof.  Note also that the assumptions made in the above lemma would need to be strengthened slightly to utilize such a result, formulating assumption \eqref{CCV-pointwise bound} on $[-4,0]$ instead of $[-3,0]$.  

Using such a compactness result, and with such a strengthened assumption, we could assume for contradiction that no such $\gamma$ exists, then find a limit function $u_\infty$ which would have all the properties of $u$ used in this proof, but with the additional property that $A_{1/2}(t) = \emptyset$ for all $t \in [-3,0]$.  

In the case $s > 1$, because $H^{s/2}$ functions cannot have jump discontinuities, we would immediately conclude that $S = \emptyset$ as well.  This would of course simplify the proof significantly.  However, because compactness is used, the proof is no longer constructive.  

In chapter~\ref{ch:kinetic}, this combined approach in used to prove the isoperimetric inequality.  To illustrate the method, we apply this method below to prove a toy lemma.  

%-------------%-------------%-------------%-------------%-------------%-------------%-------------%-------------
%-------------%-------------%  Proof with compactness   %-------------%-------------%-------------%-------------
%-------------%-------------%-------------%-------------%-------------%-------------%-------------%-------------

\begin{lemma}[Isoperimetric Inequality]
Let $\delta>0$ be a constant, $s \in (0,2)$, $d \in \N$, and let $X$ be a compact subset of $L^2([-3,0]\times\R^d)$.  Then, there exists $\gamma > 0$, and $\lambda \in (0,1)$, depending only on $\delta$, $d$, and $s$, such that for any solution $u:[-3,0] \times \R^d \to \R$ of \eqref{eq:CCV-main} satisfying $u \in X$, 
\[ u(t,x) \leq 1 + \psi_\lambda(x) \qquad \textrm{on } [-3,0]\times\R^d, \]
\[ \abs{\{u < \phi_0\} \cap (-3,-2)\times B_1} \geq |B_1|, \]
and
\[ \abs{\{u > \phi_1\} \cap (-2,0)\times \R^d} \geq \delta, \]
we have
\[ \abs{\{\phi_0 < u < \phi_1\} \cap (-3,0)\times \R^d} \geq \gamma. \]
\end{lemma}

\begin{proof}
Suppose that the lemma is false.  Then there must exist a sequence of solutions $u_k \in X$ which satisfy
\begin{equation} \label{CCV-pointwise bound k}
u(t,x) \leq 1 + \psi_{1/k}(x) \qquad \textrm{on } [-3,0]\times\R^d, 
\end{equation}
\[ \abs{\{u < 0\} \cap (-3,-2)\times B_1} \geq |B_1|, \]
and
\[ \abs{\{u > 1 + (1/k)F\} \cap (-2,0)\times B_3} \leq \delta, \]
but
\[ \abs{\{\phi_0 < u < 1 + (1/k)F\} \cap (-3,0)\times B_3} \leq 1/k. \]

Since $u_k \in X$ a compact set, we have $u_k \to u_\infty$ converging up to a subsequence in $L^2([-3,0]\times\R^d)$.  Notice that for any $(t,x) \in [-3,0]\times B_3$ either $u_\infty(t,x) \leq 1+F(x)$ or else $u_\infty(t,x) = 1$.  

%Consider the sequence $w_k := (u_k - \phi_0)_+$.  By \eqref{CCV-pointwise bound k}, we know $w_k = (u_k - \psi_1 - F)_+$ because both vanish on $|x|>3$.  Using \eqref{CCV-energy inequality generic}, we see that $\ddt \int $

Define the following set-valued functions:
\begin{align*}
A_0(t) &= \{x\in B_1: u_\infty(x,t) \leq 0\} \\
%A_{1/2}(t) &= \{x\in \R^d: 1+F < u_\infty(x,t) < 1 \} \\
A_1(t) &= \{x\in B_3: u_\infty(x,t) = 1 \}.
\end{align*}
Clearly we have $|A_0(t)| + |A_1(t)| \geq |B_1|$ for all $t$.  By assumption, we must have $\int_{-3}^{-2} |A_0(t)|dt \geq |B_1|$ and $\int_{-2}^0 |A_1(t)|dt \geq \delta$.  

By Chebyshev's inequality, for any $\alpha$ sufficiently small we have
\begin{equation} \label{CCV-big 99percent of the time k}\begin{aligned}
\abs{[-3,-2] \cap \{|A_0| > \alpha\}} &\geq 0.99, \\
|[-2,0] \cap \{|A_1| > \alpha\}| &\geq 1.99.
\end{aligned} \end{equation}
Choose one such $\alpha$, small enough that
\begin{equation} \label{CCV-asssumption on gamma_0 k}
2 \alpha< |B_1|.  
\end{equation}

\vskip.3cm

Let $\theta > 0$ and define
\[ \phi_{1/2}:= \psi_\theta + \theta F. \]
Because $\phi_0 \leq \phi_{1/2}$ for $k$ sufficiently large, we know that $u_\infty \geq \phi_{1/2}$ only when $u_\infty = 1$.  Therefore
\[ \paren{u_{\infty}-\phi_{1/2}}_+ = - \theta F \indic{A_1}. \]

One can calculate that
\[ \iint [\phi_{1/2}(y)-\phi_{1/2}(x)]^2 K(t,x,y) = C \theta^2  \]
%\[ \leq C \iint \indic{B_3}(x) [\psi_{1/3}(y)-\psi_\lambda(x)]^2 K(t,x,y) + \iint [2\lambda F(x)-2 \lambda F(y)]^2 K(t,x,y) \leq C \lambda^2 \]
and that for any $k$ sufficiently large
\begin{align*} 
\int \paren{u_k-\phi_{1/2}}_+^2 &\leq C \int (\theta F)^2
\\ &= C \theta^2.
\end{align*}

Therefore, by the energy inequality \eqref{CCV-energy inequality generic}, for all $k$ sufficiently large we have
\[ \ddt \int \paren{u_k-\phi_{1/2}}_+ \,dx \leq C \theta^2 \]
which implies, since $u_k \to u_\infty$ in $L^2$,
\begin{equation} \label{CCV-derivative bound limit} 
\ddt \int \paren{u_\infty-\phi_{1/2}}_+ \,dx \leq C \theta^2.
\end{equation}
In the same manner, we know
\begin{equation} \label{CCV-energy inequality good term k}
- \int_{-3}^0 B\bracket{\paren{u_k-\phi_{1/2}}_+, \paren{u_k-\phi_{1/2}}_-} \,dt \leq C \theta^2.
\end{equation}

On the other hand, for any fixed $t \in [-3,0]$ and $k$ sufficiently large, the good term is bounded below
\begin{align*} 
-B\bracket{\paren{u_k-\phi_{1/2}}_+, \paren{u_k-\phi-{1/2}}_-} &= 2 \iint \paren{u_k-\phi_{1/2}}_+(x) \paren{u_k-\phi_{1/2}}_-(y) K(x,y)
\\ & \geq \iint\limits_{x,y \in B_3} \paren{u_k-\phi_{1/2}}_+(x) \paren{u_k-\phi_{1/2}}_-(y) 6^{-d-2s}
\\ &= C \int_{B_3} \paren{u_k-\phi_{1/2}}_+ \,dx \int_{B_3} \paren{u_k-\phi_{1/2}}_- \,dx.  
\end{align*}

Because $u_k \to u_\infty$ strongly in $L^2$, we can take a limit of this bilinear expression and obtain, using \eqref{CCV-energy inequality good term k},
\begin{equation}
\int_{-3}^0 \paren{ \int_{B_3} \paren{u_\infty-\phi_{1/2}}_+ \,dx \int_{B_3} \paren{u_\infty-\phi_{1/2}}_- \,dx} \leq C \theta^2.  
\end{equation} 

Note that
\[ \int_{B_3} \paren{u_\infty-\phi_{1/2}}_+ = \theta \int_{A_1} |F| \geq \theta f(|A_1|) \]
for $f$ some explicit monotone-increasing function, and similarly
\[ \int_{B_3} \paren{u_\infty-\phi-{1/2}}_- \geq (1-\theta) \int_{A_0} |F| \geq f(|A_0|). \]

Therefore, we have $\int_{-3}^0 f(|A_1|) f(|A_0|) \,dt \leq C \theta$ and we conclude that $|A_1|$ and $|A_0|$ are not often simultaneously large, especially as $\theta$ decreases.  Explicitly, if we define a set of singular times
\[ S := \{ t \in [-3,0]: |A_1|>\alpha \textrm{ and } A_0 > \alpha\} \]
then, in little-o notation,
\begin{equation} \label{CCV-small measure of overlap limit} 
|S| = o(\theta).
\end{equation}


\vskip.3cm

Now we consider the function
\[ H(t) := \int \paren{u_\infty(t,x) - \phi_{1/2}(x)}_+^2 \,dx = \theta^2 \int_{A_1(t)} F(x)^2 \,dx. \]
Recall from \eqref{CCV-derivative bound limit} that $H'(t) \leq C \theta^2$.  

Because $|A_0| + |A_1| \geq |B_1|$ and $F(x)^2 \in [0,1]$ for all $x$,
\[ \theta^2[|B_1| - |A_0(t)|] \leq H(t) \leq \theta^2 |A_1(t)|. \]
In particular, if $|A_1| < \alpha$ then $H(t) \leq \theta^2 \alpha$, and if $|A_0| < \alpha$ then $H(t) \geq \theta^2[|B_1|-\alpha]$.  By assumption \eqref{CCV-asssumption on gamma_0 k}, $\theta^2 \alpha < \theta^2 [|B_1|-\alpha]$.  

Therefore $H(t)$ can only be in the range $(\theta^2 \alpha, \theta^2[|B_1|-\alpha])$ when $|A_0(t)|$ and $|A_1(t)|$ are both greater than $\alpha$, i.e. when $t \in S$.  As we have seen in \eqref{CCV-small measure of overlap limit}, this set shrinks to zero measure as $\theta \to 0$.  

By \eqref{CCV-big 99percent of the time}, there exist points $s \in [-3,-2]$ such that $|A_0(s)|>\alpha$ and hence $H(s) \leq $

By \eqref{CCV-big 99percent of the time k} and \eqref{CCV-small measure of overlap limit}, there exists a point $s \in [-3,-2]$ such that $H(s) \leq \theta^2 \alpha$ because
\[ \{H \leq \theta^2 \alpha \} \supseteq \{|A_1| \leq \alpha|\} \supseteq [-3,-2] \cap \{|A_0| > \alpha\} \cap S^\complement. \]
Simalarly, there exists a point $t \in [-2,0]$ such that $H(t) \geq \theta^2 [|B_1|-\alpha]$.  Because $H' \leq C \lambda^2$ by \eqref{CCV-derivative bound}, we know by elementary calculus that
\[ |S| \geq \abs{\{H \in (\theta^2 \alpha, \theta^2[|B_1|-\alpha)} \geq \frac{\theta^2[|B_1|-\alpha] - \theta^2 \alpha }{\sup H'} \geq \frac{\theta^2[|B_1| - 2\alpha] }{C \theta^2} = \bar{C}. \]
The constant $\bar{C}$ does not depend on $\theta$, so this cannot possibly be true for all $\theta$.  This is a contradiction.  

\end{proof}