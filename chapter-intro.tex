\chapter{Introduction} \label{ch:intro}

\section{Hilbert's Nineteenth Problem} \label{sec:intro-19}

In 1900, Hilbert laid out a list of 23 problems that he felt would guide the future of mathematics, similar to our modern Millennium Prize problems.  The nineteenth such problem was a foundational question in the calculus of variations: 
\begin{problem}
Let $\Omega \subseteq \R^d$, and let $X \subseteq L^2(\Omega)$ be the functions satisfying some boundary condition.  

Let $F:\R^d \to \R$ be a smooth uniformly convex function with bounded derivative.  Are elements of $X$ for which the energy
\[ E(u) := \int F(\grad u) \,dx, \]
is minimized necessarily smooth?
\end{problem}
It was already known that, for example, minimizers of $\int |\grad u|^2$ are harmonic functions and hence analytic.  The hypothesis was that for uniformly convex Lagrangians, which in particular satisfy $F(\xi) \approx |\xi|^2$, minimizers would be similarly regular.  

The Euler-Lagrange equation for such an energy $E$ is (using Einstein summation convention and subscripts representing derivatives) $\del_i F_i(\grad u) = 0$.  By taking a derivative of this expression in the $j\ith$ direction,  we obtain
\begin{equation} \label{intro-differentiated Euler Lagrange} \del_i \paren{ F_{ik}(\grad u) \del_k u_j} = 0. \end{equation}

At this point we can consider the equation \eqref{intro-differentiated Euler Lagrange} as a linear equation in $u_j$.  Simply define $A_{ik}(x) := F_{ik}(\grad u(x))$ and consider
\[ \div( A \grad u_j) = 0. \]
This technique is known as ``freezing'' the coefficient or ``freezing'' the equation.  Note that it is distinct from ``linearizing'' the equation, which involves expanding the equation around a given point using the first derivative and Taylor's theorem, though that term is often used colloquially to refer to both procedures.  

Because $F$ is smooth by assumption, the coefficient matrix $A$ will have the same amount of regularity as $\grad u$.  For example, if $u \in C^{k,\alpha}$ for some $k \in \N_{>0}$ and $\alpha \in (0,1)$, then $A \in C^{k-1,\alpha}$ by the chain rule and basic composition laws for H\"{o}lder continuous functions.  Moreover, we know a priori that there exists a constant $\lambda \in (0,1)$ so $\lambda |\xi|^2 \leq \xi^\transpose A \xi \leq \lambda\n |\xi|^2$ just by the uniform coercivity assumption on $F$.  

These facts give us a simple strategy to prove the existence of smooth solutions.  
\begin{enumerate}
\item By the direct method, a minimizer of $E$ can be constructed in $H^1(\Omega)$.   \\
\item Using the De Giorgi-Nash-Moser Theorem and the a priori bounds on $A$, we can show that any weak solution $w \in H^1(\Omega)$ of $\div(A \grad w) = 0$ is necessarily H\"{o}lder continuous in $C^\alpha(\Omega)$.  \\
\item By Schauder's Theorem, if $A \in C^{k,\alpha}$ and $u \in C^{k+1,\alpha}$ then $w = u_j \in C^{k+2,\alpha}$ as well (c.f. Gilbarg and Trudinger \cite{GiTr})
\end{enumerate}
By induction, we find that $u \in C^{k,\alpha}$ for all $k \in \N$.  

Historically, the De Giorgi-Nash-Moser theorem was the most difficult step to prove, and so Hilbert's Nineteenth Problem was first proven in De Giorgi's 1957 paper \cite{DG} (independently also by Nash \cite{Na.dg} in the same year, and later by Moser \cite{Mo.dg} in 1960).  

\section{The De Giorgi Method} \label{sec:intro-DG outline}

The method of De Giorgi was first applied to this elliptic problem, but in fact the core concept has wide-ranging applications.  We will explore the method below, using the toy model of an inhomogeneous parabolic equation.  

Let $d \geq 1$ a dimension and $\lambda \in (0,1)$ a coercivity parameter, let $A:\R_+\times\R^d \to \R^{d\times d}$ a matrix satisfying $\lambda |\xi|^2 \leq \xi^\transpose A(t,x) \xi \leq \lambda\n |\xi|^2$ for all $\xi \in \R^d$, and let $f:\R_+\times\R^d \to \R$ a scalar forcing term.  Suppose that $u \in L^2(\R_+;H^1(\R^d)) \cap L^\infty(\R_+; L^2(\R^d))$ satisfies the parabolic equation
\begin{equation} \label{eq:intro-parabolic}
\del_t u = \div(A \grad u) + f \qquad \textrm{ for } (t,x) \in \R_+\times\R^d.
\end{equation}

\vskip.3cm

The first step is to derive an energy inequality.  Assuming $u$ solves \eqref{eq:intro-parabolic} in the sense of distributions, we can formally multiply the equation by a test function of the form $\varphi(t,x) (u-k)_+$ for $\varphi$ a smooth cutoff function (i.e. which is identically one on some space-time region $Q_1$ and identically zero outside of some region $Q_2$) and $k$ an arbitrary constant.  This new equality can be manipulated by standard integration by parts and H\"{o}lder-type estimates into an energy inequality.  Of course test functions must be smooth, and in general we cannot assume that $(u-k)_+$ is smooth.  In this parabolic case, we can generally assume that a solution $u$ exists in $L^2(H^1)$, which is sufficient to justify this formal calculation.  In general, one may need to take the energy inequality so derived as an a priori assumption, and construct solutions which satisfy the energy inequality but not necessarily the regularity assumption needed to justify it (as in \cite{StVa.sqg}).  

The energy inequality, in the parabolic case, will have the general form
\[ \int_{A} (u(t,\cdot)-k)_+^2 \,dx + \int_s^t \int_A \abs{\grad (u-k)_+}^2 \,dxdt \hskip8cm\]
\[ \hskip5cm \leq C \bracket{\int_B (u(s,\cdot)-k)_+^2 \,dx + \norm{f}_{L^p} \paren{\int_s^t \int_B (u-k)_+^{p^*} \,dxdt}^{1/p^*}} \]
where $A \subsetneq B$ are two bounded open sets with $A$ compactly contained in $B$, $s < t$ are two times, $1/p + 1/p^* = 1$, and the constant $C$ depends on the ellipticity constant $\lambda$ and the distance between $A$ and $B^\complement$.  This is called an energy inequality because the energy $\int(u-k)_+^2$ at a time $t$ is smaller than the energy at an earlier time $s$, where the dissipation term $\int |\grad (u-k)_+|^2$ causes the energy to decrease or dissipate, and the source term corresponding to $f$ puts more energy into the system, allowing the energy to increase.  Note that the energy in a small region $A$ is bounded by the energy in a larger region $B$, to account for energy that travels through space and enters through the boundary.  

Though this form of the energy inequality makes the dynamics of the system clear, a more useful form is to consider intervals in time rather than points in time.  For $A$ a bounded open region in space and $[t,s]$ a time-interval, consider some parameter $\eps > 0$ and define $A_\eps$ to be the $\eps$-envelope around $A$ (the points within distance $\eps$ of $A$).  We call $[a-\eps,b] \times A_\eps$ a parabolic envelope of $[a,b] \times A$.  Then
\begin{equation} \label{intro-energy inequality}
\sup_{t \in [a,b]} \int_A (u-k)_+^2 + \int_a^b \int_A |\grad (u-k)_+|^2 \leq C \bracket{ \int_{a-\eps}^b \int_{A_\eps} (u-k)_+^2 + \paren{\int_{a-\eps}^b \int_{A_\eps} (u-k)_+^{p^*} }^{1/p^*} }. 
\end{equation}

\vskip.3cm

Armed with this energy inequality, the next step is to prove the so-called ``first De Giorgi lemma.''  This result is sometimes called $L^2-\textrm{to}-L^\infty$ regularization. It comes in two flavors: a global-in-space variety which shows that any solution to \eqref{eq:intro-parabolic} with $L^2$ initial data is necessarily in $L^\infty$ locally in time, and a localized version which states that if $(u-k)_+$ has a sufficiently small $L^2$ norm in the parabolic envelope of some local region, then $(u-k)_+$ will satisfy an $L^\infty$ bound in that region.  For different PDE, either one of these results may or may not hold, but in general the former version is simpler so we shall concentrate on the latter version.  It is the latter version which is useful in the proof of H\"{o}lder continuity.  

Specifically, the first De Giorgi lemma states that 
\begin{lemma}[First De Giorgi Lemma] \label{thm:intro-DG1}
There exists a constant $\delta_0$ such that, for $u$ a solution to \eqref{eq:intro-parabolic} on $[0,2]\times B_2$ with $\norm{f}_{L^p([0,2]\times B_2)} \leq 1$ for some $p$ sufficiently large, if
\[ \iint_{[0,2]\times B_2} (u-0)_+^2 \,dxdt \leq \delta_0 \]
then $u(t,x) \leq \frac{1}{2}$ for $(t,x) \in [1,2] \times B_1$.  
\end{lemma}

The proof of this lemma is by recursion.  A sequence of functions $u_k := (u-\frac{1-2^{-k}}{2})_+$ and regions $Q_k := [1-2^{-k},2]\times B_{1+2^{-k}}$ are considered, and the goal is to prove that if 
\[ \iint_{Q_0} u_0^2 = \iint_{[0,2]\times B_2} (u-0)_+^2 \leq \delta_0 \]
then 
\[ \lim_{k \to \infty} \iint_{Q_k} u_k^2 = \iint_{[1,2]\times B_1} (u-1)_+^2 = 0. \]
To accomplish this, one constructs a recursive inequality comparing $\iint_{Q_k} u_k$ to $\iint_{Q_{k-1}} u_{k-1}$.  

The main ingredient of this recursive inequality is the energy inequality \eqref{intro-energy inequality}, which compares the $L^\infty(L^2) \cap L^2(H^1)$ norm of $u_k$ on $Q_k$ to the $L^2 \cap L^{p^*}$ norm of $u_k$ on $Q_{k-1}$.  

By a variant of Sobolev's inequality, the $L^\infty(L^2) \cap L^2(H^1)$ norm of $u$ on the left-hand-side of \eqref{intro-energy inequality} controls the $L^q$ norm of $u$, for some $q > 2$.  As compared to the $L^2$ norm, the $L^\infty(L^2)$ norm has greater control on integrability in time, and the $L^2(H^1)$ norm has greater control on the integrability in space.  By interpolation we obtain an improvement in both time and space:
\[ \norm{u_k}_{L^q(Q_{k-1})}^2 \leq C \bracket{\norm{u_k}_{L^2(H^1)}^2 + \norm{\grad u_k}_{L^2(Q_{k-1})}^2 }. \]  

We now want to see that the right-hand-side of the energy inequality, the $L^2 \cap L^{p^*}$ norm, is controlled by the $L^q$ norm of $u_{k-1}$ on $Q_{k-1}$, with the same $q$ as on the left-hand-side.  This is true because $u_{k-1}$ is bounded below on the support of $u_k$, meaning that in particular $u_k^a \leq 2^{(k+1)(b-a)}u_{k-1}^b$ for any $0 \leq a < b$. We have the non-linear bound, assuming $p^* \leq q$,
\[ \iint_{Q_{k-1}} u_k^2 + \paren{\iint_{Q_{k-1}} u_k^{p^*} }^{1/p^*} \leq \iint_{Q_{k-1}} u_{k-1}^q + \paren{ \iint_{Q_{k-1}} u_{k-1}^q }^{1/p^*}. \]
Notice that the exponent on $u_{k-1}$ is always $q$, but the exponent on the integral itself varies.  Combining this with the energy inequality and our bound on the left-hand-side of the energy inequality, we obtain
\[ \norm{u_k}_{L^q(Q_{k-1})} \leq C^k \bracket{ \norm{u_{k-1}}_{L^q(Q_{k-1})}^{q/2} + \norm{u_{k-1}}_{L^q(Q_{k-1})}^{q/(2p^*)} }. \]
So long as the exponents $\frac{q}{2}$ and $\frac{q}{2 p^*}$ are strictly greater than 1, this inequality is superlinear.  In that case, if the initial value of the sequence is sufficiently small, the limit will be zero.  This is sufficient to prove the lemma.  

This proof method works because the energy inequality bounds a first-order norm (meaning a norm involving first derivatives) by an zeroth-order norm (meaning a norm involving no derivatives).  For a generic function, inequalities such as Sobolev's embedding show that zeroth-order norms are controlled by first-order norms, but the opposite is not true for generic functions  This is a very strong property of solutions to \eqref{eq:intro-parabolic} and other parabolic equations.  After reducing the order of the energy inequality by applying Sobolev embedding, we find that the $L^q$ norm is bounded by an $L^2$ norm and an $L^{p^*}$ norm.  Recall that for a generic function on a bounded domain, Lebesgue norms with large exponents bound Lebesgue norms with smaller exponents.  However, assuming $p$ is sufficiently large (specifically $1/p + 2/q < 1$), we have $2,p^* < q$ so the reduced energy inequality bounds a large-exponent Lebesgue norm by two small-exponent Lebesgue norms.  It is therefore not surprising that $L^2-\textrm{to}-L^\infty$ regularization is possible.  When applying the De Giorgi method to a given PDE, among the first questions one must ask is whether the natural exponent $q$ is larger than any exponents which may appear on the right-hand-side of the energy inequality.  

\vskip.3cm

The next step in the De Giorgi argument is to prove the second De Giorgi lemma, also known as the isoperimetric inequality.  
\begin{lemma}[Second De Giorgi Lemma] \label{thm:intro-DG2}
There exists a constant $\mu_0 > 0$ such that, for $u$ a solution to \eqref{eq:intro-parabolic} on $[-1,4]\times B_3$ with $\norm{f}_{L^p([-1,4]\times B_3)} \leq 1$ for some $p$ sufficiently large, if
\begin{equation} \label{intro-DG2 assumption Linfty} u(t,x) \leq 2 \qquad \forall (t,x) \in [-1,4] \times B_3 \end{equation}
and
\begin{equation} \label{intro-DG2 assumption above} \abs{\{u \geq 1\} \cap [2,4] \times B_2} \geq \delta_0 \end{equation}
and
\begin{equation} \label{intro-DG2 assumption below} \abs{\{u \leq 0\} \cap [0,4] \times B_2} \geq \frac{\abs{[0,4]\times B_2}}{2} \end{equation}
then
\begin{equation} \label{intro-DG2 assumption middle} \abs{\{0 < u < 1\} \cap [0,4] \times B_2} \geq \mu_0. \end{equation}
\end{lemma}

The isoperimetric lemma is a quantitative version of the claim ``solutions to \eqref{eq:intro-parabolic} cannot have jump discontinuities.''  The assumption \eqref{intro-DG2 assumption Linfty}, together with the energy inequality, will give us a regularity estimate for the solution $u$.  In practice, this assumption will be ensured by an application of the first De Giorgi lemma.  The claim \eqref{intro-DG2 assumption above} uses the same $\delta_0$ as in the statement of the first De Giorgi lemma \ref{thm:intro-DG1}.  The intention is that, if \eqref{intro-DG2 assumption above} fails to be satisfied, then the first lemma can be applied to some translation of $(u-1)_+$.  

Note that the assumption \eqref{intro-DG2 assumption Linfty} must be satisfied on a larger region $[-1,4]\times B_3$ than the rest of our assumptions.  This is so that, by the energy inequality \eqref{intro-energy inequality}, $(u-0)_+$ will be in $L^2([0,4]; H^1(B_2))$.  If we could say that, given assumption \eqref{intro-DG2 assumption Linfty}, $\norm{(u-0)_+}_{H^1([0,4] \times B_2)}$ were uniformly bounded, meaning it is regular in both time and space, and we could set $\mu_0 = 0$, then the lemma would be trivial, simply because a function in $H^1$ cannot have a jump discontinuity.  

Assuming still that $\norm{(u-0)_+}_{H^1([0,4] \times B_2)}$ uniformly bounded, even for $\mu_0 > 0$ we could easily prove the result by contradiction.  Take a sequence of solutions $u_k$ satisfying \eqref{intro-DG2 assumption Linfty}-\eqref{intro-DG2 assumption below} but such that
\[ \abs{\{0 < u_k < 1\} \cap [0,4] \times B_2} \leq \frac{1}{k}. \]
This sequence $u_k$ would be uniformly bounded in $H^1$, and so it would have a strong $L^2$ limit $u_\infty$.  But $u_\infty \in H^1$ and $u_\infty$ has a jump discontinuity, which gives us our contradiction.  

Unfortunately, it is generally not the case that $(u-0)_+$ is $H^1$-regular in time.  The proof of Lemma~\ref{thm:intro-DG2} relies on showing that the sequence $u_k$ has enough uniform regularity in time (derived still from the assumption \eqref{intro-DG2 assumption Linfty}) so that the strong-$L^2$ limit $u_\infty$ exists, and cannot have a jump discontinuity.  In general, it is easier to bound $\del_t u_\infty$ from above than to bound it from below.  This is why assumption \eqref{intro-DG2 assumption above} is phrased on the time-interval $[2,4]$ rather than $[0,4]$: to guarantee that a jump discontinuity in $u_\infty$ will exist such that $\del_t u_\infty \geq 0$ in the sense of distributions.  

The actual technique for showing regularity-in-time is highly dependent on the specific PDE in question, and so it is not useful to give a more detailed outline.  

\vskip.3cm

Once the first and second De Giorgi lemmas are proven, the proof of H\"{o}lder continuity is typically similar across different applications of the method.  One merely needs to apply the first and second lemmas to various translated and scaled copies of $u$.  It is necessary therefore that the equation \eqref{eq:intro-parabolic} be symmetric under a large family of transformations.  In particular, we will consider transformations of the form $\bar{u}(t,x) := C + \alpha u(t,x)$ for \emph{possibly negative} constants $C, \alpha \in \R$.  

%For any $u$ solving \eqref{eq:intro-parabolic} with coercivity $\lambda$ and source term $f$, the function $\bar{u}(t,x) := a + b u(t_0 + c^2 t, x_0 + c x)$, for constants $a,b, t_0 \in \R$ and $c \in \R^+$ and $x_0 \in \R^d$, will solve \eqref{eq:intro-parabolic} with coercivity $\lambda$ and source term $\bar{f}(t,x) := b c^2 f(t_0 + c^2 t, x_0 + c x)$.  Note that $\norm{\bar{f}}_{L^p} = b c^{2 - (d+2)/p} \norm{f}_{L^p}$, which means $\norm{\bar{f}}_{L^p} \leq \norm{f}_{L^p}$ so long as $p$ is large and $c$ is small relative to $b$.  Therefore, if we wish to investigate the behavior of a solution in some small region, we can simply apply 

%It is, therefore, essential that the first and second lemmas are formulated in a way that is invariant under translations and scalings, in particular under transformations of the form $\bar{u}(t,x) := C + \alpha u(t,x)$ for \emph{possibly negative} constants $C, \alpha \in \R$.  

A common intermediate step between the De Giorgi lemmas and the proof of H\"{o}lder continuity is known as the oscillation lemma.  Note that some sources use the name ``second De Giorgi lemma'' to refer to the oscillation lemma, rather than the isoperimetric inequality.  

The oscillation of a function over a set $S$ is defined by 
\[ \underset{S}{\osc} f := \sup_{x \in S} f(x) - \inf_{x \in S} f(x). \]
The oscillation lemma then states that the oscillation of a solution $u$ to \eqref{eq:intro-parabolic} over a space-time region $Q$ is bounded by a constant times the oscillation of the same $u$ over a parabolic envelope of that region.  We will present the lemma in a more rigid formulation, for clarity of presentation:
\begin{lemma}[Oscillation Lemma]
There exists a constant $\lambda_0 > 0$ such that, for $u$ a solution to \eqref{eq:intro-parabolic} on $[-1,4]\times B_3$ with $\norm{f}_{L^p([-1,4]\times B_3)} \leq \lambda_0$ for $p$ sufficiently large, if
\[ -2 \leq u(t,x) \leq 2 \qquad \forall (t,x) \in [-1,4] \times B_3 \]
then 
\[ \underset{[3,4] \times B_1}{\osc} u \leq 4-\lambda_0. \]
\end{lemma}

The oscillation will either decrease from above or below, depending whether the measure $\abs{\{u \leq 0\} \cap [0,4]\times B_2\}}$ is more or less than $\frac{\abs{[0,4]\times B_2}}{2}$.  We can assume without loss of generality that it is greater, and otherwise we can simply apply the following argument to $-u$.  

Consider the sequence of functions $u_0 = u$ and $u_k := 2u_{k-1} - 2$.  Note that each $u_k$ will solve \eqref{eq:intro-parabolic} (with source term $2^k f$) and satisfy assumptions \eqref{intro-DG2 assumption Linfty} and \eqref{intro-DG2 assumption below} of the second De Giorgi lemma.  

Assume that for some $k_0$, the function $u_{k_0}$ satisfies the assumption \eqref{intro-DG2 assumption above}.  In particular, because of the way the sequence $u_k$ is constructed, this means that $u_k$ satisfies \eqref{intro-DG2 assumption above} for all $0 \leq k \leq k_0$.  Therefore we can apply the second De Giorgi lemma and find that 
\[ \abs{\{0 \leq u_k \leq 1\} \cap [0,4] \times B_2} \geq \mu_0 \qquad \forall 0 \leq k \leq k_0. \]
Because each of these $k_0$ sets are disjoint by construction, and because they are all contained in $[0,4]\times B_2$ which has finite measure, we find that our assumption on $k_0$ cannot hold for $k_0 = \abs{[0,4]\times B_2} / \mu_0$.  For this $k_0$, we conclude that $u_{k_0}$ does not satisfy \eqref{intro-DG2 assumption above}.  

We can now apply the first De Giorgi lemma to $u_{k_0} - 1$ and conclude that $u_{k_0}(t,x) \leq 3/2$ for $(t,x) \in [3,4]\times B_1$, or equivalently $u(t,x) \leq 2 - 2^{-(k_0+1)}$.  

\vskip.3cm

Recall that a function $g$ is H\"{o}lder continuous with exponent $\alpha$ at a point $y_0$ if and only if
\[ \underset{|y-y_0| \leq r}{\osc} g(y) \leq C r^\alpha. \]
Therefore, by applying the oscillation lemma to dilations of $u$, we can easily conclude that $u$ is H\"{o}lder continuous.  

Because the constant $\mu_0$ in the second De Giorgi lemma is obtained by a compactness argument, none of the constants obtained thereafter can be explicit.  Most notably, the exponent $\alpha$ is not explicit.  Therefore it is often desirable to obtain the second De Giorgi lemma by a more constructive argument, as in \cite{Gu.quantitative}.  

\section{Main Results and Outline}


%
%
%
%\begin{theorem} \label{thm:intro-DG toy theorem}
%For any $T > 0$, there exists a constant $C > 0$ and $\alpha \in (0,1)$ so that all solutions $u$ to \eqref{eq:intro-parabolic} will satisfy 
%\[ \norm{u}_{L^\infty((T,\infty)\times\R^d)} \leq C \norm{u(0,\cdot)}_{L^2(\R^d)} \]
%and
%\[ \norm{u}_{C^\alpha((2T,\infty)\times\R^d)} \leq C \norm{u}_{L^\infty((T,\infty)\times\R^d)}. \]
%
%The constants $\alpha$ and $C$ depend on $\lambda$ and on $f$ and $b$ and $d$, as well as $T$.  
%\end{theorem}
%
%We will prove this well-known theorem using the De Giorgi method.  
%
%\begin{theorem}
%Energy Inequality
%\end{theorem}
%
%\begin{proof}
%For any $s \in [-1,0]$, let $\eta_s:[-2,0]\to \R$ be given by 
%\[ \eta_s(t) = \begin{cases}
%t+2 & t \in [-2,-1], \\
%1 & t \in [-1,s], \\
%0 & t \in (s,0].
%\end{cases} \]
%Let $\varphi: \R^d\to\R$ be a smooth function which is identically 1 on $B_1$ and identically zero on the complement of $B_2$, satisfying $\norm{\grad\varphi}_{L^\infty} \leq 2$.  
%Since $u \in L^2(H^1)$ by assumption, we can multiply \eqref{eq:intro-parabolic} by the test function $\eta_s(t) \varphi(x)^2 (u-k)_+$ to obtain
%\begin{equation} \label{intro-integrate equation against (u-k)_+} 
%\frac{1}{2} \iint [(u-k)_+ \varphi]^2 (-\eta_s') + \iint \eta_s \grad [\varphi^2 (u-k)_+] A \grad(u-k)_+ = \iint \eta_s \varphi^2(u-k)_+ b \cdot \grad(u-k)_+ + \iint \eta_s \varphi^2 (u-k)_+ f. 
%\end{equation}
%
%The left hand side is bounded
%\begin{equation} \label{intro-right hand side of energy} \begin{aligned} 
%\frac{1}{2} \iint [(u-k)_+ \varphi]^2 (-\eta_s') + \iint \eta_s \grad [\varphi^2 (u-k)_+] A \grad(u-k)_+ &\geq \frac{1}{2}\int [\varphi (u-k)_+]^2|_{t=s} \,dx + 2 \iint \eta_s \grad[\varphi (u-k)_+] A \grad[\varphi (u-k)_+] - \frac{1}{2} \int_{-2}^{-1} \int [(u-k)_+ \varphi]^2 - 2 \iint \eta_s \grad[\varphi (u-k)_+] A \grad[\varphi] (u-k)_+ 
%\\ &\geq \frac{1}{2}\int [\varphi (u-k)_+]^2|_{t=s} \,dx + \iint \eta_s \grad[\varphi (u-k)_+] A \grad[\varphi (u-k)_+] - \frac{1}{2} \int_{-2}^{-1} \int [(u-k)_+ \varphi]^2 - \iint \eta_s (u-k)_+^2 \grad\varphi A \grad\varphi
%\\ & \geq \frac{1}{2}\int_{B_1} (u-k)_+^2|_{t=s} \,dx + \lambda \int_{-1}^s \int_{B_1} |\grad(u-k)_+|^2 - \frac{1}{2} \int_{-2}^{-1} \int_{B_2} (u-k)_+^2 - \lambda\n 2 \int_{-2}^0 \int_{B_2} (u-k)_+^2.
%\end{aligned} \end{equation}
%
%Define $p^*$ such that $\frac{1}{p} + \frac{1}{p^*} = 1$.  The right hand side of \eqref{intro-integrate equation against (u-k)_+} is bounded
%\begin{equation*} \begin{aligned}
%\iint \iint \eta_s \varphi^2 (u-k)_+ f 
%&\leq \norm{f}_{L^p} \paren{ \int_{-2}^0 \int_{B_2} (u-k)_+^{p^*} }^{1/p^*}.
%\end{aligned} \end{equation*}
%By combining this with \eqref{intro-integrate equation against (u-k)_+} and \eqref{intro-right hand side of energy}, we obtain
%\[ \frac{1}{2}\int_{B_1} (u-k)_+^2|_{t=s} \,dx + \lambda \int_{-1}^s \int_{B_1} |\grad(u-k)_+|^2 \leq \frac{1}{2} \int_{-2}^{-1} \int_{B_2} (u-k)_+^2 + \lambda\n 2 \int_{-2}^0 \int_{B_2} (u-k)_+^2 + \norm{f}_{L^p} \paren{ \int_{-2}^0 \int_{B_2} (u-k)_+^{p^*} }^{1/p^*}. \]
%Taking the supremum of the left hand side over all $s \in [-1,0]$, and recalling that all of our integrands are non-negative, we obtain that in particular
%\[ \sup_{t \in [-1,0]} \int_{B_1} (u-k)_+^2 \,dx + \int_{-1}^0 \int_{B_1} |\grad(u-k)_+|^2 \leq C\paren{\lambda, \norm{f}_{L^p}} \paren{ \int_{-2}^{0} \int_{B_2} (u-k)_+^2 + \paren{ \int_{-2}^0 \int_{B_2} (u-k)_+^{p^*} }^{1/p^*} }. \]
%
%
%\end{proof}
%
%\begin{theorem}
%There exists a constant $\delta_0$ such that any solution which satisfies
%\[ \int_{-2}^0 \int_{B_2} (u)_+^2 \,dxdt \leq \delta_0 \]
%will also satisfy $u \leq 1$ on $[-1,0]\times B_1$.  
%\end{theorem}
%
%\begin{proof}
%We consider the sequence of truncated functions $u_k := (u-1 + 2^{-k})_+$ and the sequence of regions $Q_k := [-1-2^{-k},0] \times B_{1 + 2^{-k}}$.  Note that $u_0 = (u)_+$ and, in the limit, $u_\infty = (u-1)_+$, while $Q_0 = [-2,0]\times B_2$ and, in the limit, $Q_\infty = [-1,0]\times B_1$. 
%
%For each truncated function, we consider the energy
%\[ \E_k = \sup_{[-1-2^{-k},0]} \int u_k^2 + \iint_{Q_k} \abs{\grad u_k}^2. \]
%
%We wish to show that $\E_k$ bounds some Lebesgue norm of $u_k$, using a combination of Sobolev embedding and Riesz-Thorin interpolation.  To that end, consider the function $m(t) := \int_{B_{1+2^{-k}}} u_k$ (the spatial average of $u_k$), and note that $0 \leq m(t) \leq \sqrt{|B_2|} \E_k^{1/2}$ by H\"{o}lder's inequality.  
%
%By the Sobolev embedding, we know that 
%\begin{align*} 
%\int_{-1-2^{-k}}^0 \paren{ \int_{B_{1+2^{-k}}} u_k^{\frac{2d}{d-2}} }^{\frac{d-2}{2d}} 
%&\leq \norm{ \grad u_k}_{L^1([-1-2^{-k},0]; L^2(B_{1+2^{-k}}))} + \int_{-1-2^{-k}}^0 \paren{ \int_{B_{1+2^{-k}}} m^{\frac{2d}{d-2}} }^{\frac{d-2}{2d}}
%\\ &\leq \sqrt{1+2^{-k}} \norm{ \grad u_k}_{L^2(Q_k)} + m(t) (1+2^{-k}\int_{-1-2^{-k}}^0 \paren{ \int_{B_{1+2^{-k}}} 1 }^{\frac{d-2}{2d} }
%\\ &\leq C \E_k^{1/2}.
%\end{align*}
%The constant $C$ here depends only on the dimension.  The quantity $\E_k^{1/2}$ then controls both the $L^1\paren{ L^{2d/(d-2)} }$ and the $L^\infty\paren{L^2}$ norm of $u_k$ over the region $Q_k$.  By interpolation, we conclude that for any $q_1 \in (0,1)$, $q_2 \in (1/2-1/d, 1/2)$ satisfying
%\[ \frac{2}{d} \frac{1}{q_1} + 2 \frac{1}{q_2} = 1, \]
%the quantity $\E_k^{1/2}$ also controls the $L^{q_1}\paren{L^{q_2}}$ norm of $u_k$ on $Q_k$.  This equation represents the line segment between $(1,1/2-1/d)$ and $(0,1/2)$.  Simply because $1/2 - 1/d$ is less than $1/2$, we know that there must exist a point on this line segment such that $q_1 = q_2$ called $(q,q)$ with $q = 2 + \frac{2}{d}$.  
%
%We conclude that 
%\begin{equation} \label{intro-lower bound on energy E_k}
%\iint_{Q_k} u_k^q \leq \E_k^{q/2}. 
%\end{equation}
%Note that, whatever the value of $d$, we have $q/2 > 1$.  We have shown that the energy $\E_k$ has superlinear control on some Lebesgue norm of $u_k$.  
%
%\vskip.3cm
%
%Now that we have used Sobolev embedding to bound $\E_k$ from below, we can use the energy inequality to bound it from above.  First, define the indicator functions $\chi_k := \indic{u_k > 0}$ so that $u_k = \chi_k \paren{u_{k-1} - 2^{-k-1} }$ and 
%\[ 2^{-k-1} \chi_k = \chi_k u_{k-1} - u_k \leq u_{k-1}. \]
%We have by the energy inequality \eqref{eq:intro-energy inequality}, assuming $p^* \leq q$, 
%\begin{align*} 
%\E_k &\leq C 2^k \bracket{ \iint_{Q_{k-1}} u_k^2 + \paren{ \iint_{Q_{k-1}} u_k^{p^*} }^{1/p^*} }
%\\ &= C^k \bracket{ \iint_{Q_{k-1}} u_k^2 \chi_k^{q-2} + \paren{ \iint_{Q_{k-1}} u_k^{p^*} \chi_k^{q - p*} }^{1/p^*} }
%\\ &\leq C^k \bracket{ \iint_{Q_{k-1}} u_{k-1}^2 (2^{k+1} u_{k-1})^{q-2} + \paren{ \iint_{Q_{k-1}} u_{k-1}^{p^*} (2^{k+1} u_{k-1})^{q - p*} }^{1/p^*} }
%\\ &\leq C^k \bracket{ \iint_{Q_{k-1}} u_{k-1}^q + \paren{ \iint_{Q_{k-1}} u_{k-1}^q }^{1/p^*} }.
%\end{align*}
%By combining this inequality with \eqref{intro-lower bound on energy E_k}, we finally arrive at a recursive inequality for the sequence $\E_k$:
%\[ \E_k \leq C^k \bracket{ \E_{k-1}^{q/2} + \E_{k-1}^{q/(2p^*)} }. \]
%
%\end{proof}
%
%This is too long.  
%
%
%\section{Research Statement Intro}

The remaining chapters of this dissertation will present will present various problems to which the De Giorgi method can be applied.  Chapters \ref{ch:hamjac}, \ref{ch:kinetic}, \ref{ch:SQG}, and \ref{ch:conservation} are based on the works \cite{StVa.hamjac}, \cite{St.hypo}, \cite{StVa.sqg}, and \cite{St.shocks} respectively, with only minor modifications.  

Early applications of the De Giorgi method were to equations which were either elliptic or parabolic, or nonlinear equations which can be reduced to an elliptic or parabolic form.  One defining feature of such equations is that the regularity of the solutions is primarily driven by the second-order term; lower order terms can be viewed as perturbations.  However, the method is capable of tackling equations which are not elliptic at all.  In chapter~\ref{ch:hamjac}, we apply the method to a problem in which the regularization is driven by a first-order term $|\grad u|^p$.  This equation contains a second-order term which is not only non-elliptic, it acts as an accumulative (rather than dissipative) force and hence an obstacle to regularization.  In chapter~\ref{ch:kinetic}, we apply the method to a hypoelliptic kinetic equation which has a Laplacian-like term $(-\Laplace_v)^s f$ acting only in the $v$-direction and has no terms at all which directly regularize in the $x$-direction.  Regularization in the $x$-direction is caused not by an elliptic term but by a hyperbolic mixing term.  In chapter~\ref{ch:SQG}, we apply the method to an equation which has both a first-order dissipation term $(-\Laplace)^{1/2} \theta$ and also a first-order transport term.  The transport term can ultimately be viewed as a perturbation, but only after a very precise analysis of the energy inequality.  Finally, in chapter~\ref{ch:conservation} we consider a different kind of energy method, the relative entropy method, and apply this to a hyperbolic conservation equation.  In appendix~\ref{ch:CCVrewrite}, we elaborate on an alternative proof of the second De Giorgi lemma which applies to nonlocal equations.  The technique shown there is already known in the literature, but is intended to be an entry-level survey.  

%My research interests are in the extension of techniques from the regularity theory of elliptic PDE to a wide variety of problems.  I have applied elliptic methods to problems which are highly degenerate.  I favor energy-based approaches, and am particularly well versed in the De Giorgi method, an elliptic method for proving continuity of weak solutions.  Most of the problems I have studied are related to fluid mechanics.  I have worked on projects involving nonlocal operators, boundary effects, kinetic equations, and a mixing of divergence-form and non-divergence-form methods, among many other hurdles.   

%My greatest area of recent interest is in SQG on bounded domains. SQG is a fluid equation derived from geophysics, and is now well understood in homogeneous domains, but since it is nonlocal the boundary behavior presents serious complications to be overcome.  This nonlinear problem has seen intense recent study from the group of Peter Constantin.  I describe this problem and my recent preprint on the topic \cite{StVa.sqg} in section \ref{sec:sqg} below.  

%I have also published papers on two types of degenerate elliptic problem: hypoelliptic equations \cite{St.hypo} which are mixed elliptic and hyperbolic type, and superquadratic Hamilton-Jacobi equations \cite{StVa.hamjac} which are essentially-first-order equations and not elliptic at all.  Both of these papers utilize the De Giorgi method originating in pure elliptic theory.  While both of these problems had been tackled previously using elliptic methods, my new contrbutions required a melding of these techniques with other ideas from different fields as well as an improved understanding of the De Giorgi method itself.  The problems and my contributions are described in sections \ref{sec:hypo} and \ref{sec:hamjac} below.  

%My most recent result is on the stability theory of shocks to viscous conservation equations \cite{St.shocks}.  This result is very different from my other results, since it deals with stability as opposed to regularity, but is still approached from the perspective of energy.  This problem and my contribution is discussed in section \ref{sec:shocks} below.  


\vskip.3cm
%\section{On superquadratic first-order Hamilton-Jacobi equations} \label{sec:hamjac}
In chapter~\ref{ch:hamjac} we consider a Hamilton-Jacobi equation with superquadratic growth in its first-order term.  

Hamilton-Jacobi equations are a class of highly nonlinear PDE.  They are typically studied as a non-divergence form problem (using techniques such as Perron's method and maximum principles), as opposed to divergence form(using techniques such as distributional solutions and energy estimates), because they are not typically in the form of an Euler-Lagrange equation.  The class of Hamiltonian that we will study are of the form
\[ \del_t u = H(x,u,\grad u, D^2u) \approx -|\grad u|^p + \textrm{error terms} \]
where $p > 2$ is a constant.  Equations of this form appear in the work of Schwab \cite{Sc.hamjac}, in the study of homogenization for stochastic optimal control problems; taking the homogenous limit requires compactness, which comes in the form of a uniform regularity estimate.  The first order term $|\grad u|^p$ has a regularization effect, as proven by Cardaliaguet \cite{Ca}.  % and also by Chan and Vasseur \cite{ChVa} using the De Giorgi method.  
For $p > 2$, the first order term will dominate even when the ``error'' includes certain second order terms, as had been proven using probabilistic methods and comparison with sub- and supersolutions (\cite{CaCa}, \cite{CaRa}), including the most comprehensive result by Cardaliaguet and Silvestre \cite{CaSi.hamjac}.  Using a modification of the De Giorgi method, the case without second-order error was tackled by Chan and Vasseur in \cite{ChVa}.  The case with second-order error will be addressed in chapter~\ref{ch:hamjac} in which we will prove the following:
\begin{theorem}[c.f. Theorem \ref{th:main}, first proven in \cite{StVa.hamjac}]
Let $\Omega$ an open subset of $\R^n$ and $[0,T]$ a time interval.  Let $p > 2$ and $\Lambda \geq 1$ be constants, and $f \in L^q$ with $q$ sufficiently large, and $A:[0,T]\times \Omega \to \R^{n\times n}$ satisfying $\abs{\xi^\intercal A(t,x) \xi} \leq \Lambda |\xi|^2$ for all vectors $\xi$.  
%Let $H$ be a Hamiltonian $H:[0,T]\times \Omega \times \R \times \R^n \times \R^{n \times n} \to \R$ satisfying
%\[ \frac{1}{\Lambda} |v|^p + - f(x,t) - \div(A\grad 

Then if $u:[0,T]\times\Omega \to \R$ is a weak solution to
\[ \del_t u + \frac{1}{\Lambda} |\grad u|^p \leq f + \div(A \grad u), \]
\[ \del_t u + \Lambda |\grad u|^p \geq -\Lambda + m^-(D^2 u) \]
for $m^-$ the least-negative-eigenvalue function, $u$ will be H\"{o}lder continuous uniformly on any compact subset of $(0,T] \times \Omega$.  Its modulus of continuity will depend on $p$, $\Lambda$, and $\norm{u}_{L^\infty}$.  
\end{theorem}
Note that $A$ is not assumed non-negative-definite.  

This problem initially seems ill-suited for De Giorgi's method.  It is nonlinear in an essential way and has no corresponding energy functional, which is why most previous investigations used maximum principles instead of energy methods which typically involve some form of linearization.  De Giorgi method is an elliptic method, meaning it is typically driven by the coercivity of the second order term.  This problem is not only non-elliptic, its second order term is actually a major obstacle to regularity.  By tackling this problem using De Giorgi's method, however, we are able to expand the class of allowable errors beyond what was known in the literature, specifically allowing for discontinuous coefficients on the second order term and for source terms which are unbounded from above.  

One major technical hurdle is that, unlike in the work of Chan and Vasseur \cite{ChVa}, the Caccioppoli inequality that underlies the De Giorgi method will only be valid in regions where $u$ is large.  Another hurdle is the non-linearity of the equation; normally with De Giorgi we break nonlinear equations into a coupled linear (or at least variational) system and only treat these simpler equations, but in this problem the superquadratic growth is essential to the regularization.   Lastly, as a consequence of the nonlinearity, the regularization from below happens backwards in time.  In order to obtain the regularization that we desire, we must construct subsolutions which transport the regularity forwards in time and apply a maximum principle, thus mixing divergence-form and non-divergence-form methods.  




\vskip.3cm
%\section{On hypoelliptic kinetic equations} \label{sec:hypo}
In chapter \ref{ch:kinetic}, we consider a family of hypoelliptic kinetic equations.  

Hypoelliptic equations are a class of degenerate elliptic equations with mixed elliptic and hyperbolic features.  In particular, they include certain kinetic equations of the form
\begin{equation} \label{eq:hypo} \del_t f + v\cdot \grad_x f - Q(f) = \sigma \end{equation}
for $f(t,x,v)$ a function of time, space and velocity, $Q$ an elliptic operator in the velocity variable, and $\sigma$ a source term.  The idea is that such equations are elliptic in some variables ($v$) and hyperbolic in others ($x$), but due to mixing ($v \cdot \grad_x$) regularization occurs in all variables.  

General hypoelliptic equations were studied by Kolmogorov and by H\"{o}rmander \cite{Ho} as early as the 1930s.  They originally considered only smooth solutions, but more recently a Sobolev $H^s$ theory developed in the form of the study of averaging lemmas (\cite{Ag}, \cite{GoLiPeSe}).  An averaging lemma states that if the kinetic derivative of a function $\bracket{\del_t + v\cdot\grad_x} f$ is bounded in the Sobolev sense, then the velocity averages $\int f \,dv$ are regular.  The theory of averaging lemmas had been developed into a full Sobolev hypoelliptic theory by Bouchut \cite{Bo}, for $Q$ a fractional Laplacian, and that theory in turn was used by Golse, Imbert, Mouhot and Vasseur \cite{GoImMoVa} to prove H\"{o}lder continuity for $Q$ a second order local elliptic operator in $v$.  

I particularly studied the case of $Q$ a uniformly elliptic singular integral operator.  Such equations occur in the study of astrophysics to model particles interacting with a plasma (\cite{Ka}, \cite{Go.physics}, \cite{LaKe}, \cite{MeRo}).  They are also of interest for their relationship to the Boltzmann equation in the absence of the Grad cutoff assumption.  Authors including Silvestre, Mouhot, Imbert and others (\cite{ImSi}, \cite{ImMoSi}, \cite{Mo}, \cite{Si}, \cite{HeSnTa}, \cite{HeSn}) have recently been building an elliptic theory of the Boltzmann and Landau equations.  In a landmark paper of this project, Silvestre and Imbert were able to prove H\"{o}lder continuity for a class of equations including both the Boltzmann equation and general uniformly elliptic singular integral operators as subcases.  

In chapter \ref{ch:kinetic} we study \eqref{eq:hypo} with $Q$ an integral operator in $v$ given by
\[ Q(f)(t,x,v) := \int K(t,x,v,w) [f(t,x,w) - f(t,x,v)] \,dw, \]
with $K$ symmetric in $v$ and $w$ and $\kappa\n \leq K(t,x,v,w)|v-w|^{n+2s} \leq \kappa$ for some $s \in (0,1)$ and constant $\kappa$, and $\sigma$ an $L^p$ function with finite but large $p$.  
We are able to show that solutions are $C^\alpha$ regular even with merely $H^s$ initial data:
\begin{theorem}[c.f. Theorem \ref{thm:main}, frst proven in \cite{St.hypo}]
Let $\Omega$ and open subset of $\R^n$ and $[0,T]$ a time interval.  Let $s \in (0,1)$ and $\kappa \geq 1$ be constants and $Q$ as described above.  

Then there exists $p^* < \infty$ so that for any function $\sigma \in L^p \cap L^2([0,T] \times \Omega \times \R^n)$ and any weak solution $f \in L^\infty([0,T] \times \Omega \times \R^n) \cap L^2([0,T]\times\Omega; H^s(\R^n))$ to \eqref{eq:hypo} will be H\"{o}lder continuous in space and time on any compact subset of $(0,T]\times \Omega$.  The modulus of continuity depends only on $\sigma$ and the $L^\infty$ norm of the initial data.  
\end{theorem}
The class of operators we consider is a special case of that considered by Imbert and Silvestre \cite{ImSi}, who combined De Giorgi techniques and a Krylov approach to obtain regularity.  Using the theory of averaging lemmas, we are able to apply solely energy-based techniques for a simplified proof and to allow for unbounded source terms.  This work is inspired by that of Golse, Imbert, Mouhot and Vasseur \cite{GoImMoVa}, but requires a different approach due to the nonlocality of $Q$.  In particular, we cannot use the full regularity result of Bouchut and need to work directly with a more standard averaging lemma \cite{Be}.  We are able to obtain $L^2 \to L^p$ ($p>2$) regularization from the regularization of averages, and then use the De Giorgi method to turn this into $C^\alpha$ regularization.  

%In the future, I wish to extend the nonlinear theory of hypoelliptic operators by considering cases in which the coercivity of the elliptic operator depends on the solution.  Nonlinearities of this type are common in physical problems, and in the local case they are sometimes ammenable to techniques similar to those I used in \cite{StVa.hamjac} (see section \ref{sec:hamjac}).  This problem is therefore at the intersection of nonlinear and nonlocal techniques with which I am very familiar, but whose interaction poses significant new challenges.  


\vskip.3cm
%\section{On well-posedness of SQG on bounded domains} \label{sec:sqg}
In chapter \ref{ch:SQG} we consider the Surface Quasi-Geostrophic equation (SQG) on a bounded domain.  

The SQG equation is a special case of the equations describing large-scale atmospheric and oceanic currents (\cite{Pe}, \cite{Ch}).  In addition to its physical importance, SQG is of a mathematical form with interesting commonalities to the 3D Euler equations (\cite{CoMaTa}), which explains its widespread study in pure and applied fluid mechanics.  The form of SQG that we study, with critical dissipation, is
\begin{equation} \label{eq:sqg} \del_t \theta + \bracket{\grad^\perp (-\Delta)^{-1/2} \theta} \cdot \grad \theta + (-\Delta)^{1/2} \theta = 0. \end{equation}
Note that $(-\Delta)^{1/2}$ and $(-\Delta)^{-1/2}$ are both nonlocal operators.  

Well-posedness for SQG on $\R^2$ has been known since 2010 (\cite{CaVa}, \cite{KiNaVo}, \cite{CoVi}), with multiple proofs from various perspectives.  Physically motivated by, for example, air currents near land-sea boundaries, a few authors (\cite{Kr.landsea}, \cite{NoVa.bounded}, \cite{NoVa.solutions}) have considered SQG on bounded domains.  There are multiple ways to define the boundary behavior of such a system, however, so several different models have been proposed.  In any case, the behavior of the nonlocal operator near the boundary complicates the analysis and demands new techniques.  

Recently, Constantin and Ignatova \cite{CoIg.fraclap} proposed a new model for SQG on bounded domains (specifically, defining $(-\Delta)^{1/2}$ as the spectral square-root of the Laplacian with homogeneous Dirichlet boundary conditions).  They and Nguyen have published several papers on the topic, studying existence, uniqueness, regularity, and convergence for the equation with varying strengths of dissipation (\cite{CoIg.fraclap}, \cite{CoIg.sqg}, \cite{CoIgNg}, \cite{CoNg}, \cite{CoNg.strong}).  In particular, Constantin and Ignatova \cite{CoIg.sqg} showed that, for sufficiently regular initial data, solutions to the critical SQG are smooth in the interior of the domain.  In said paper they identify boundary regularity, and specifically H\"{o}lder continuity up to the boundary, as a difficult open problem and an important step in the analysis of this equation.  

In chapter \eqref{ch:SQG} we obtain H\"{o}lder continuity up to the boundary:
\begin{theorem}[c.f. Theorem \ref{thm:main continuity}, first proven in \cite{StVa.sqg}]
Let $\Omega$ a bounded open subset of $\R^2$ with smooth boundary, $[0,T]$ a time interval, and $\theta:[0,T] \times \Omega \to \R$ a weak solution to critical SQG \eqref{eq:sqg} in $L^\infty(0,T; L^4(\Omega)) \cap L^2(0,T; H_0^1(\Omega))$ with $L^2$ initial data $\theta_0$.  

Then $\theta$ is H\"{o}lder continuous in time and space on $[\eps,T]\times \bar{\Omega}$ for any $\eps > 0$.  The modulous of continuity depends only on the domain and $\norm{\theta_0}_{L^2}$.  
\end{theorem}
Critical SQG has a dissipation term which is regularizing, and a transport term which has the potential to be deregularizing.  Since the problem is critical, they are ostensibly equal in strength, so it is difficult to predict how solutions will behave.  The transport term is particularly difficult to control near the boundary because the Dirichlet boundary conditions on $(-\Delta)$ are not translation invariant, and hence the commutator $[\grad, (-\Delta)^{1/2}]$, is singular in this region.  %but most of the control on this commutator blows up near the boundary because the Dirichlet boundary conditions on $(-\Delta)$ are not translation invariant and hence will not commute with differentiation.  
This proof is inspired by the previous work of Caffarelli and Vasseur \cite{CaVa} on the global SQG to consider weaker norms in which the transport term may be bounded.  In the case of global SQG, BMO is strong enough to constrain the regularity of the solution but weak enough that $\grad^\perp(-\Laplace)^{-1/2}$ is a bounded operator, 
while for the bounded-domain case, it was necessary to define a sophisticated and novel Banach space (somewhat analogous to $B_{\infty,\infty}^0(\Omega)$) adapted to this specific problem.  This space is based on a generalized analogue of Littlewood-Paley theory (first studied by Iwabuchi, Matsuyama, and Taniguchi \cite{IMT.besov}, \cite{IMT.bilinear}, \cite{IMT.schrodinger}) in order to distinguish the low/high frequencies present in the commutator.  
%a modification of Littlewood-Paley theory (first studied by Iwabuchi, Matsuyama, and Taniguchi \cite{IMT.besov}, \cite{IMT.bilinear}, \cite{IMT.schrodinger}) was necessary to define a Banach space adapted to this specific problem (a space somewhat analogous to $B_{\infty, \infty}^0(\Omega)$).  
The proof also utilizes a mixed Eulerian-Lagrangian approach with a moving reference frame adapted to counteract the transport term.  Though utilized previously in the case of $\R^2$, this approach presents new difficulties in this case because, from a Lagrangian perspective, the domain is time-dependent.  

%I am currently investigating whether solutions to SQG are differentiable (as opposed to merely continuous) up to the boundary.  There is a heuristic argument that $C^1$ should follow readily from $C^\alpha$ (because the $L^\infty \Rightarrow C^\alpha$ result makes the equation ``subcritical''), but certain calculations with toy models imply $C^\alpha$ may be the optimal regularity.  If global $C^1$ regularization can be proven, long-time existence of classical solutions will follow.  


\vskip.3cm
%\section{On stability of shocks} \label{sec:shocks}
In chapter \ref{ch:conservation} we consider the stability of shocks to a conservation law.  

A shock is a special kind of traveling wave solution to a conservation law, i.e. of the form $s(x-\sigma t)$ for $\sigma$ constant.  A 1D inviscid shock is discontinuous at a single point and is constant, with two distinct values, on either side of that discontinuity; a viscous shock is a smoother approximation thereof.  There is a significant literature devoted to the stability of shocks in both the $L^1$ and $L^2$ norms, so we shall concentrate on the case of scalar case.  Many results give stability only for small perturbations (\cite{FrSe}, \cite{Kr.entropy}, \cite{IlOl}).  Large perturbation $L^2$ stability of small inviscid shocks has been achieved by Vasseur and his group (e.g. \cite{SeVa}, \cite{LeVa}, \cite{Le}) using the relative entropy method first introduced by DiPerna and Dafermos \cite{Da} to study stability of Lipschitz solutions.  $L^2$ stability will generally only hold up to a shift which depends on the solution:
\[ \forall \textrm{ sol'n } u, \exists \gamma \textrm{ s.t. } \ddt \int |u(t,x) - s(x-\gamma(t))|^2 \,dx \leq 0. \]

In the dissipative case, meaning conservation laws of the form
\begin{equation} \label{eq:cons} \del_t u + \div A(u) = \nu \Laplace \eta'(u), \end{equation}
most of the time it is necessary to consider instead a weighted $L^2$ norm with a weight function $a$, called $L^2$-type stability.  A recently developed relative-entropy technique has been able to obtain $L^2$-type stability with $a$ arbitrarilly close to 1 in the $L^\infty$ sense.  This technique has been used in the case of 1D dissipative systems \cite{KaVa.navier} (including 1D Navier-Stokes) and 1D scalar equations with near-constant dissipation \cite{KaVa.burgers} (e.g. $\eta'(u)=u$).  

We prove in chapter \ref{ch:conservation} that 1D dissipative scalar conservation laws with uniformly convex flux and a nonlinear viscosity are $L^2$-type stable for sufficiently small shocks, independent of the dissipative parameter $\nu$:
\begin{theorem}[Theorem 1 in \cite{St.shocks}]
Let $\eta, A: \R \to \R$ be any uniformly convex functions with continuous third derivatives at 0. Then there exists $\eps_0 > 0$ such that for any $\nu > 0$ and $\eps \in (0,\eps_0]$ the following holds:

If $s:\R \to \R$ is a shock solution to \eqref{eq:cons} with $\norm{s}_{L^\infty} \leq \eps$, then there exists a weight function $a:\R \to (0,2)$ such that $s$ is $L^2$-type stable up to a shift.  Moreover, $\norm{a-1}_{L^\infty}$ can be made arbitrarilly small by taking $\eps$ sufficiently small.  
\end{theorem}
Because this result holds independently of the strength of dissipation $\nu$, the result will apply also to vanishing viscosity limit solutions to the equivalent hyperbolic conservation law.  

We use the relative entropy method.  We are able to handle a wider variety of nonlinear viscosities by utilizing $\eta$ (the function appearing in the dissipative term) as the entropy.  This proof only uses one entropy for each equation, which is important if a technique is expected to generalize to systems; conservation systems typically only have a single entropy.  

As in previous $L^2$-type estimates, we break up the solution $u$ into a part which is $L^\infty$ close to $s$ and an error term which may be large in $L^\infty$.  The close part is handled similarly to the existing literature, while for the error term we need to make careful use of the relationship between the dissipative term and the mass density of the derivative of the weight function $a$.  

%This result is part of a rich ongoing program.  Open problems include expansions to multiple independent variables and more general flux and viscosities (particularly with higher growth rates at infinity).  This technique has already been applied to systems of conservation laws in \cite{KaVa.navier}, so this new result may translate also to the system case.  There is a possibility to handle multiple shocks in superposition, though in the viscous case there is significant work to be done understanding how shocks interact as they cross one another; recall that the relative motion of shocks is complex due to the Rankine-Hugoniot condition and the presence of artificial shifts.  


\vskip.3cm

\textbf{Notation. } Throughout this work, $C$ will represent arbitrary constants which may change from line to line.  The function space $\Ctest$ contains smooth functions with compact support.  We will use the notation $(x)_+ := \max(0,x)$.  When the parentheses are ommited, the subscript~$+$ is merely a label. 
