\chapter{Introduction}

%\documentclass[a4paper, 11pt]{article}
%
%\usepackage{fancyhdr}
%\usepackage{lastpage}
%\usepackage[dvips]{color}
%\usepackage{amsmath}
%\usepackage{amssymb}
%\usepackage{amsthm}
%%\usepackage{amscd}
%\usepackage{amsfonts}
%\usepackage[alphabetic]{amsrefs}
%\usepackage{graphicx}
%\usepackage{hyperref}
%
%%\raggedright
%\setlength{\parindent}{1cm}
%
%%\advance\oddsidemargin-0.65in
%\oddsidemargin-.15in
%\topmargin-.5in
%\textheight9.5in
%\textwidth6.5in
%\newcommand\bb[1]{\mbox{\em #1}}
%\def\baselinestretch{1.05}
%%\pagestyle{empty}
%
%\newcommand{\hsp}{\hspace*{\parindent}}
%\definecolor{gray}{rgb}{0.4,0.4,0.4}
%%\definecolor{gray}{rgb}{1.0,1.0,1.0}
%
%%%%%%%%%%%%%%%%%%%%%%%%%%%%%%%%%%%%%%%%%%
%\newcommand{\R}{\mathbb{R}}
%\newcommand{\N}{\mathbb{N}}
%\newcommand{\Z}{\mathbb{Z}}
%\newcommand{\Q}{\mathbb{Q}}
%\newcommand{\C}{\mathbb{C}}
%\newcommand{\Prj}{\mathbb{P}}
%\newcommand{\F}{\mathbb{F}}
%\newcommand{\A}{\mathbb{A}}
%\newcommand{\Four}{\mathcal{F}}
%\newcommand{\T}{\mathbb{T}}
%\newcommand{\E}{\mathbb{E}}
%%%%%%%%%%%%%%%%%%%%%%%%%%%%%%%%%%%%%%%%%%
%\newcommand{\floor}[1]{\left\lfloor #1 \right\rfloor}
%\newcommand{\ceil}[1]{\left\lceil #1 \right\rceil}
%\newcommand{\chevron}[1]{\langle #1 \rangle}
%\newcommand{\norm}[1]{\left\lVert#1\right\rVert}
%\newcommand{\paren}[1]{\left( #1 \right)}
%\newcommand{\bracket}[1]{\left[ #1 \right]}
%\newcommand{\abs}[1]{\left\lvert #1 \right\rvert}
%%%%%%%%%%%%%%%%%%%%%%%%%%%%%%%%%%%%%%%%%%
%\DeclareMathOperator{\id}{id}
%\DeclareMathOperator{\convex}{conv}
%\DeclareMathOperator{\image}{Im}
%\DeclareMathOperator{\im}{Im}
%\DeclareMathOperator{\coker}{coker}
%\DeclareMathOperator{\supp}{supp}
%\DeclareMathOperator{\trace}{tr}
%\DeclareMathOperator{\lspan}{span}
%\DeclareMathOperator{\conv}{conv} % stands for conv, as in convex hull
%\DeclareMathOperator{\Int}{int} % stands for int, as in interior of a set
%\DeclareMathOperator{\sign}{sign}
%\DeclareMathOperator{\ran}{ran}
%\DeclareMathOperator{\rank}{rank}
%%\DeclareMathOperator{\dim}{dim}
%\newcommand{\dom}{\operatorname{dom}}
%\newcommand{\cod}{\operatorname{cod}}
%\newcommand{\Hom}{\operatorname{hom}}
%\newcommand{\Ob}{\operatorname{Ob}}
%\newcommand{\cl}{\operatorname{cl}}
%%%%%%%%%%%%%%%%%%%%%%%%%%%%%%%%%%%%%%%%%%%
%\newcommand{\del}{\partial}
%\newcommand{\pvec}[2]{\frac{\partial #1}{\partial #2}}
%\newcommand{\grad}{\nabla}
%\newcommand{\ddt}{\frac{d}{dt}}
%\renewcommand{\div}{\operatorname{div}}
%\newcommand{\Laplace}{\Delta}
%%%%%%%%%%%%%%%%%%%%%%%%%%%%%%%%%%%%%%%%%%%
%\newcommand{\into}{\hookrightarrow}
%\newcommand{\onto}{\twoheadrightarrow}
%\newcommand{\isom}{\cong}
%\newcommand{\rest}{{\upharpoonright}}
%\newcommand{\weakly}{\rightharpoonup}
%%%%%%%%%%%%%%%%%%%%%%%%%%%%%%%%%%%%%%%%%%%
%\newcommand{\ith}{^\mathrm{th}}
%\newcommand{\n}{^{-1}}
%\newcommand{\eps}{\varepsilon}
%%%%%%%%%%%%%%%%%%%%%%%%%%%%%%%%%%%%%%%%%%%
%\newcommand{\indic}[1]{\chi_{\{#1\}}}
%%%%%%%%%%%%%%%%%%%%%%%%%%%%%%%%%%%%%%%%%%%
%\newtheorem*{theorem}{Theorem}
%
%\begin{document}
%\thispagestyle{fancy}
%\lhead{}
%\rhead{}
%\renewcommand{\headrulewidth}{0pt} 
%\renewcommand{\footrulewidth}{0pt} 
%\fancyfoot[C]{\footnotesize \textcolor{gray}{http://www.ma.utexas.edu/$\sim$lstokols}} 
%
%%\pagestyle{myheadings}
%%\markboth{Sundar Iyer}{Sundar Iyer}
%
%\pagestyle{fancy}
%\lhead{\textcolor{gray}{\it Logan Stokols}}
%\rhead{\textcolor{gray}{\thepage/\pageref{LastPage}}}
%%\rhead{\thepage}
%%\renewcommand{\headrulewidth}{0pt} 
%%\renewcommand{\footrulewidth}{0pt} 
%%\fancyfoot[C]{\footnotesize http://www.stanford.edu/$\sim$sundaes/application} 
%%\ref{TotPages}
%
%
%
%%\vspace*{0.1cm}
%\begin{center}
%{\LARGE \bf RESEARCH STATEMENT}\\
%\vspace*{0.1cm}
%{\normalsize Logan Stokols (lstokols@math.utexas.edu)}
%\end{center}
%%\vspace*{0.2cm}


My research interests are in the extension of techniques from the regularity theory of elliptic PDE to a wide variety of problems.  I have applied elliptic methods to problems which are highly degenerate.  I favor energy-based approaches, and am particularly well versed in the De Giorgi method, an elliptic method for proving continuity of weak solutions.  Most of the problems I have studied are related to fluid mechanics.  I have worked on projects involving nonlocal operators, boundary effects, kinetic equations, and a mixing of divergence-form and non-divergence-form methods, among many other hurdles.   

My greatest area of recent interest is in SQG on bounded domains. SQG is a fluid equation derived from geophysics, and is now well understood in homogeneous domains, but since it is nonlocal the boundary behavior presents serious complications to be overcome.  This nonlinear problem has seen intense recent study from the group of Peter Constantin.  I describe this problem and my recent preprint on the topic \cite{StVa.sqg} in section \ref{sec:sqg} below.  

I have also published papers on two types of degenerate elliptic problem: hypoelliptic equations \cite{St.hypo} which are mixed elliptic and hyperbolic type, and superquadratic Hamilton-Jacobi equations \cite{StVa.hamjac} which are essentially-first-order equations and not elliptic at all.  Both of these papers utilize the De Giorgi method originating in pure elliptic theory.  While both of these problems had been tackled previously using elliptic methods, my new contrbutions required a melding of these techniques with other ideas from different fields as well as an improved understanding of the De Giorgi method itself.  The problems and my contributions are described in sections \ref{sec:hypo} and \ref{sec:hamjac} below.  

My most recent result is on the stability theory of shocks to viscous conservation equations \cite{St.shocks}.  This result is very different from my other results, since it deals with stability as opposed to regularity, but is still approached from the perspective of energy.  This problem and my contribution is discussed in section \ref{sec:shocks} below.  

\section{On well-posedness of SQG on bounded domains} \label{sec:sqg}
The Surface Quasi-Geostrophic equation (SQG) is a special case of the equations describing large-scale atmospheric and oceanic currents (\cite{Pe}, \cite{Ch}).  In addition to its physical importance, SQG is of a mathematical form with interesting commonalities to the 3D Euler equations (\cite{CoMaTa}), which explains its widespread study in pure and applied fluid mechanics.  The form of SQG that I studied, with critical dissipation, is
\begin{equation} \label{eq:sqg} \del_t \theta + \bracket{\grad^\perp (-\Delta)^{-1/2} \theta} \cdot \grad \theta + (-\Delta)^{1/2} \theta = 0. \end{equation}
Note that $(-\Delta)^{1/2}$ and $(-\Delta)^{-1/2}$ are both nonlocal operators.  

Well-posedness for SQG on $\R^2$ has been known since 2010 (\cite{CaVa}, \cite{KiNaVo}, \cite{CoVi}), with multiple proofs from various perspectives.  Physically motivated by, for example, air currents near land-sea boundaries, a few authors (\cite{Kr.landsea}, \cite{NoVa.bounded}, \cite{NoVa.solutions}) have considered SQG on bounded domains.  There are multiple ways to define the boundary behavior of such a system, however, so several different models have been proposed.  In any case, the behavior of the nonlocal operator near the boundary complicates the analysis and demands new techniques.  

Recently, Constantin and Ignatova \cite{CoIg.fraclap} proposed a new model for SQG on bounded domains (specifically, defining $(-\Delta)^{1/2}$ as the spectral square-root of the Laplacian with homogeneous Dirichlet boundary conditions).  They and Nguyen have published several papers on the topic, studying existence, uniqueness, regularity, and convergence for the equation with varying strengths of dissipation (\cite{CoIg.fraclap}, \cite{CoIg.sqg}, \cite{CoIgNg}, \cite{CoNg}, \cite{CoNg.strong}).  In particular, Constantin and Ignatova \cite{CoIg.sqg} showed that, for sufficiently regular initial data, solutions to the critical SQG are smooth in the interior of the domain.  In said paper they identify boundary regularity, and specifically H\"{o}lder continuity up to the boundary, as a difficult open problem and an important step in the analysis of this equation.  

In \cite{StVa.sqg} I obtain H\"{o}lder continuity up to the boundary:
\begin{theorem}[Theorem 1.1 in \cite{StVa.sqg}]
Let $\Omega$ a bounded open subset of $\R^2$ with smooth boundary, $[0,T]$ a time interval, and $\theta:[0,T] \times \Omega \to \R$ a weak solution to critical SQG \eqref{eq:sqg} in $L^\infty(0,T; L^4(\Omega)) \cap L^2(0,T; H_0^1(\Omega))$ with $L^2$ initial data $\theta_0$.  

Then $\theta$ is H\"{o}lder continuous in time and space on $[\eps,T]\times \bar{\Omega}$ for any $\eps > 0$.  The modulous of continuity depends only on the domain and $\norm{\theta_0}_{L^2}$.  
\end{theorem}
Critical SQG has a dissipation term which is regularizing, and a transport term which has the potential to be deregularizing.  Since the problem is critical, they are ostensibly equal in strength, so it is difficult to predict how solutions will behave.  The transport term is particularly difficult to control near the boundary because the Dirichlet boundary conditions on $(-\Delta)$ are not translation invariant, and hence the commutator $[\grad, (-\Delta)^{1/2}]$, is singular in this region.  %but most of the control on this commutator blows up near the boundary because the Dirichlet boundary conditions on $(-\Delta)$ are not translation invariant and hence will not commute with differentiation.  
I was inspired by the previous work of Caffarelli and Vasseur \cite{CaVa} on the global SQG to consider weaker norms in which the transport term may be bounded.  In the case of global SQG, BMO is strong enough to constrain the regularity of the solution but weak enough that $\grad^\perp(-\Laplace)^{-1/2}$ is a bounded operator, 
while for the bounded-domain case, it was necessary to define a sophisticated and novel Banach space (somewhat analogous to $B_{\infty,\infty}^0(\Omega)$) adapted to this specific problem.  This space is based on a generalized analogue of Littlewood-Paley theory (first studied by Iwabuchi, Matsuyama, and Taniguchi \cite{IMT.besov}, \cite{IMT.bilinear}, \cite{IMT.schrodinger}) in order to distinguish the low/high frequencies present in the commutator.  
%a modification of Littlewood-Paley theory (first studied by Iwabuchi, Matsuyama, and Taniguchi \cite{IMT.besov}, \cite{IMT.bilinear}, \cite{IMT.schrodinger}) was necessary to define a Banach space adapted to this specific problem (a space somewhat analogous to $B_{\infty, \infty}^0(\Omega)$).  
My technique also utilizes a mixed Eulerian-Lagrangian approach with a moving reference frame adapted to counteract the transport term.  Though utilized previously in the case of $\R^2$, this approach presented new difficulties in my case because, from a Lagrangian perspective, the domain is time-dependent.  

I am currently investigating whether solutions to SQG are differentiable (as opposed to merely continuous) up to the boundary.  There is a heuristic argument that $C^1$ should follow readily from $C^\alpha$ (because the $L^\infty \Rightarrow C^\alpha$ result makes the equation ``subcritical''), but certain calculations with toy models imply $C^\alpha$ may be the optimal regularity.  If global $C^1$ regularization can be proven, long-time existence of classical solutions will follow.  


\section{On hypoelliptic kinetic equations} \label{sec:hypo}
Hypoelliptic equations are a class of degenerate elliptic equations with mixed elliptic and hyperbolic features.  In particular, they include certain kinetic equations of the form
\begin{equation} \label{eq:hypo} \del_t f + v\cdot \grad_x f - Q(f) = \sigma \end{equation}
for $f(t,x,v)$ a function of time, space and velocity, $Q$ an elliptic operator in the velocity variable, and $\sigma$ a source term.  The idea is that such equations are elliptic in some variables ($v$) and hyperbolic in others ($x$), but due to mixing ($v \cdot \grad_x$) regularization occurs in all variables.  

General hypoelliptic equations were studied by Kolmogorov and by H\"{o}rmander \cite{Ho} as early as the 1930s.  They originally considered only smooth solutions, but more recently a Sobolev $H^s$ theory developed in the form of the study of averaging lemmas (\cite{Ag}, \cite{GoLiPeSe}).  An averaging lemma states that if the kinetic derivative of a function $\bracket{\del_t + v\cdot\grad_x} f$ is bounded in the Sobolev sense, then the velocity averages $\int f \,dv$ are regular.  The theory of averaging lemmas had been developed into a full Sobolev hypoelliptic theory by Bouchut \cite{Bo}, for $Q$ a fractional Laplacian, and that theory in turn was used by Golse, Imbert, Mouhot and Vasseur \cite{GoImMoVa} to prove H\"{o}lder continuity for $Q$ a second order local elliptic operator in $v$.  

I particularly studied the case of $Q$ a uniformly elliptic singular integral operator.  Such equations occur in the study of astrophysics to model particles interacting with a plasma (\cite{Ka}, \cite{Go.physics}, \cite{LaKe}, \cite{MeRo}).  They are also of interest for their relationship to the Boltzmann equation in the absence of the Grad cutoff assumption.  Authors including Silvestre, Mouhot, Imbert and others (\cite{ImSi}, \cite{ImMoSi}, \cite{Mo}, \cite{Si}, \cite{HeSnTa}, \cite{HeSn}) have recently been building an elliptic theory of the Boltzmann and Landau equations.  In a landmark paper of this project, Silvestre and Imbert were able to prove H\"{o}lder continuity for a class of equations including both the Boltzmann equation and general uniformly elliptic singular integral operators as subcases.  

In \cite{St.hypo} I studied \eqref{eq:hypo} with $Q$ an integral operator in $v$ given by
\[ Q(f)(t,x,v) := \int K(t,x,v,w) [f(t,x,w) - f(t,x,v)] \,dw, \]
with $K$ symmetric in $v$ and $w$ and $\kappa\n \leq K(t,x,v,w)|v-w|^{n+2s} \leq \kappa$ for some $s \in (0,1)$ and constant $\kappa$, and $\sigma$ an $L^p$ function with finite but large $p$.  
I was able to show that solutions are $C^\alpha$ regular even with merely $H^s$ initial data:
\begin{theorem}[Theorem 1.1 in \cite{St.hypo}]
Let $\Omega$ and open subset of $\R^n$ and $[0,T]$ a time interval.  Let $s \in (0,1)$ and $\kappa \geq 1$ be constants and $Q$ as described above.  

Then there exists $p^* < \infty$ so that for any function $\sigma \in L^p \cap L^2([0,T] \times \Omega \times \R^n)$ and any weak solution $f \in L^\infty([0,T] \times \Omega \times \R^n) \cap L^2([0,T]\times\Omega; H^s(\R^n))$ to \eqref{eq:hypo} will be H\"{o}lder continuous in space and time on any compact subset of $(0,T]\times \Omega$.  The modulus of continuity depends only on $\sigma$ and the $L^\infty$ norm of the initial data.  
\end{theorem}
The class of operators I considered is a special case of that considered by Imbert and Silvestre \cite{ImSi}, who combined De Giorgi techniques and a Krylov approach to obtain regularity.  Using the theory of averaging lemmas, I was able to apply solely energy-based techniques for a simplified proof and to allow for unbounded source terms.  My work was inspired by that of Golse, Imbert, Mouhot and Vasseur \cite{GoImMoVa}, but required a different approach due to the nonlocality of $Q$.  In particular, I could not use the full regularity result of Bouchut and had to work directly with a more standard averaging lemma \cite{Be}.  I was able to obtain $L^2 \to L^p$ ($p>2$) regularization from the regularization of averages, and then use the De Giorgi method to turn this into $C^\alpha$ regularization.  

In the future, I wish to extend the nonlinear theory of hypoelliptic operators by considering cases in which the coercivity of the elliptic operator depends on the solution.  Nonlinearities of this type are common in physical problems, and in the local case they are sometimes ammenable to techniques similar to those I used in \cite{StVa.hamjac} (see section \ref{sec:hamjac}).  This problem is therefore at the intersection of nonlinear and nonlocal techniques with which I am very familiar, but whose interaction poses significant new challenges.  


\section{On superquadratic first-order Hamilton-Jacobi equations} \label{sec:hamjac}
Hamilton-Jacobi equations are a class of highly nonlinear PDE.  They are typically studied as a non-divergence form problem (using techniques such as Perron's method and maximum principles), as opposed to divergence form(using techniques such as distributional solutions and energy estimates), because they are not typically in the form of an Euler-Lagrange equation.  The class of Hamiltonian that I studied involves superquadratic growth in the first derivative:
\[ \del_t u = H(x,u,\grad u, D^2u) \approx -|\grad u|^p + \textrm{error terms} \]
where $p > 2$ is a constant.  Equations of this form appear in the work of Schwab \cite{Sc}, in the study of homogenization for stochastic optimal control problems; taking the homogenous limit requires compactness, which comes in the form of a uniform regularity estimate.  The first order term $|\grad u|^p$ has a regularization effect, as proven by Cardaliaguet \cite{Ca}.  % and also by Chan and Vasseur \cite{ChVa} using the De Giorgi method.  
For $p > 2$, the first order term will dominate even when the ``error'' includes certain second order terms, as had been proven using probabilistic methods and comparison with sub- and supersolutions (\cite{CaCa}, \cite{CaRa}), including the most comprehensive result by Cardaliaguet and Silvestre \cite{CaSi.hamjac}.  Using a modification of the De Giorgi method, the case without second-order error was tackled by Chan and Vasseur in \cite{ChVa}, and I considered second-order errors in \cite{StVa.hamjac} resulting in the following theorem:
\begin{theorem}[Theorem 1 in \cite{StVa.hamjac}]
Let $\Omega$ an open subset of $\R^n$ and $[0,T]$ a time interval.  Let $p > 2$ and $\Lambda \geq 1$ be constants, and $f \in L^q$ with $q$ sufficiently large, and $A:[0,T]\times \Omega \to \R^{n\times n}$ satisfying $\abs{\xi^\intercal A(t,x) \xi} \leq \Lambda |\xi|^2$ for all vectors $\xi$.  
%Let $H$ be a Hamiltonian $H:[0,T]\times \Omega \times \R \times \R^n \times \R^{n \times n} \to \R$ satisfying
%\[ \frac{1}{\Lambda} |v|^p + - f(x,t) - \div(A\grad 

Then if $u:[0,T]\times\Omega \to \R$ is a weak solution to
\[ \del_t u + \frac{1}{\Lambda} |\grad u|^p \leq f + \div(A \grad u), \]
\[ \del_t u + \Lambda |\grad u|^p \geq -\Lambda + m^-(D^2 u) \]
for $m^-$ the least-negative-eigenvalue function, $u$ will be H\"{o}lder continuous uniformly on any compact subset of $(0,T] \times \Omega$.  Its modulus of continuity will depend on $p$, $\Lambda$, and $\norm{u}_{L^\infty}$.  
\end{theorem}
Note that $A$ is not assumed non-negative-definite.  

This problem initially seems ill-suited for De Giorgi's method.  It is nonlinear in an essential way and has no corresponding energy functional, which is why most previous investigations used maximum principles instead of energy methods which typically involve some form of linearization.  De Giorgi method is an elliptic method, meaning it is typically driven by the coercivity of the second order term.  This problem is not only non-elliptic, its second order term is actually a major obstacle to regularity.  By tackling this problem using De Giorgi's method, however, I was able to expand the class of allowable errors beyond what was known in the literature, specifically allowing for discontinuous coefficients on the second order term and for source terms which are unbounded from above.  

One major technical hurdle is that, unlike in the work of Chan and Vasseur \cite{ChVa}, the Caccioppoli inequality that underlies the De Giorgi method will only be valid in regions where $u$ is large.  Another hurdle is the non-linearity of the equation; normally with De Giorgi we break nonlinear equations into a coupled linear (or at least variational) system and only treat these simpler equations, but in this problem the superquadratic growth is essential to the regularization.   Lastly, as a consequence of the nonlinearity, the regularization from below happens backwards in time.  In order to obtain the regularization that we desire, we must construct subsolutions which transport the regularity forwards in time and apply a maximum principle, thus mixing divergence-form and non-divergence-form methods.  


\section{On stability of shocks} \label{sec:shocks}
A shock is a special kind of traveling wave solution to a conservation law, i.e. of the form $s(x-\sigma t)$ for $\sigma$ constant.  A 1D inviscid shock is discontinuous at a single point and is constant, with two distinct values, on either side of that discontinuity; a viscous shock is a smoother approximation thereof.  There is a significant literature devoted to the stability of shocks in both the $L^1$ and $L^2$ norms, so we shall concentrate on the case of scalar case.  Many results give stability only for small perturbations (\cite{FrSe}, \cite{Kr.entropy}, \cite{IlOl}).  Large perturbation $L^2$ stability of small inviscid shocks has been achieved by Vasseur and his group (e.g. \cite{SeVa}, \cite{LeVa}, \cite{Le}) using the relative entropy method first introduced by DiPerna and Dafermos \cite{Da} to study stability of Lipschitz solutions.  $L^2$ stability will generally only hold up to a shift which depends on the solution:
\[ \forall \textrm{ sol'n } u, \exists \gamma \textrm{ s.t. } \ddt \int |u(t,x) - s(x-\gamma(t))|^2 \,dx \leq 0. \]

In the dissipative case, meaning conservation laws of the form
\begin{equation} \label{eq:cons} \del_t u + \div A(u) = \nu \Laplace \eta'(u), \end{equation}
most of the time it is necessary to consider instead a weighted $L^2$ norm with a weight function $a$, called $L^2$-type stability.  A recently developed relative-entropy technique has been able to obtain $L^2$-type stability with $a$ arbitrarilly close to 1 in the $L^\infty$ sense.  This technique has been used in the case of 1D dissipative systems \cite{KaVa.navier} (including 1D Navier-Stokes) and 1D scalar equations with near-constant dissipation \cite{KaVa.burgers} (e.g. $\eta'(u)=u$).  

I have proven in \cite{St.shocks} that 1D dissipative scalar conservation laws with uniformly convex flux and a nonlinear viscosity are $L^2$-type stable for sufficiently small shocks, independent of the dissipative parameter $\nu$:
\begin{theorem}[Theorem 1 in \cite{St.shocks}]
Let $\eta, A: \R \to \R$ be any uniformly convex functions with continuous third derivatives at 0. Then there exists $\eps_0 > 0$ such that for any $\nu > 0$ and $\eps \in (0,\eps_0]$ the following holds:

If $s:\R \to \R$ is a shock solution to \eqref{eq:cons} with $\norm{s}_{L^\infty} \leq \eps$, then there exists a weight function $a:\R \to (0,2)$ such that $s$ is $L^2$-type stable up to a shift.  Moreover, $\norm{a-1}_{L^\infty}$ can be made arbitrarilly small by taking $\eps$ sufficiently small.  
\end{theorem}
I used the relative entropy method.  I was able to handle a wider variety of nonlinear viscosities by utilizing $\eta$ (the function appearing in the dissipative term) as the entropy.  My proof only uses one entropy for each equation, which is important if a technique is expected to generalize to systems; conservation systems typically only have a single entropy.  The result is also independent of the strength of the viscosity, meaning it is suitable for vanishing viscosity limits.  

As in previous $L^2$-type estimates, I broke up the solution $u$ into a part which is $L^\infty$ close to $s$ and an error term which may be large in $L^\infty$.  The close part is handled similarly to the existing literature, while for the error term I had to make careful use of the relationship between the dissipative term and the mass density of the derivative of the weight function $a$.  

This result is part of a rich ongoing program.  Open problems include expansions to multiple independent variables and more general flux and viscosities (particularly with higher growth rates at infinity).  This technique has already been applied to systems of conservation laws in \cite{KaVa.navier}, so this new result may translate also to the system case.  There is a possibility to handle multiple shocks in superposition, though in the viscous case there is significant work to be done understanding how shocks interact as they cross one another; recall that the relative motion of shocks is complex due to the Rankine-Hugoniot condition and the presence of artificial shifts.  


%\bibliographystyle{alpha}
%\bibliography{StokolsResearchStatement.bib}
%
%\end{document}

